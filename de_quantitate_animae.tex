%!TEX TS-program = xelatex 
%!TEX TS-options = -synctex=1 -output-driver="xdvipdfmx -q -E"
%!TEX encoding = UTF-8 Unicode
%
%  de_quantitate_animae
%
%  Created by Mark Eli Kalderon on 2016-01-09.
%  Copyright (c) 2016. All rights reserved.
%

\documentclass[12pt]{article} 

% Definitions
% \newcommand{\change}{\textcolor{blue}{\textbf{CHANGE SLIDE}}}
\newcommand\mykeywords{Augustine, vision, perception, extramission}
\newcommand\myauthor{Mark Eli Kalderon}

% Packages
\usepackage{geometry} \geometry{a4paper} 
\usepackage{url}
% \usepackage{txfonts}
\usepackage{color}
\usepackage{enumerate}
\definecolor{gray}{rgb}{0.459,0.438,0.471}
\usepackage{setspace}
% \doublespace % Uncomment for doublespacing if necessary
% \usepackage{epigraph} % optional

% XeTeX
\usepackage[cm-default]{fontspec}
\usepackage{xltxtra,xunicode}
\defaultfontfeatures{Scale=MatchLowercase,Mapping=tex-text}
\setmainfont{Hoefler Text}
\newfontfamily{\sbl}{SBL Greek}

% Bibliography
\usepackage[round]{natbib}

% Title Information
\title{Perception in \emph{De quantitate animae}}
\author{\myauthor} 
\date{} % Leave blank for no date, comment out for most recent date

% PDF Stuff
\usepackage[plainpages=false, pdfpagelabels, bookmarksnumbered, backref, pdftitle={Form Without Matter}, pagebackref, pdfauthor={\myauthor}, pdfkeywords={\mykeywords}, xetex, colorlinks=true, citecolor=gray, linkcolor=gray, urlcolor=gray, unicode=true]{hyperref} 

%%% BEGIN DOCUMENT
\begin{document}

% Title Page
\maketitle
\begin{abstract}
	\noindent Augustine is commonly interpreted as endorsing an extramission theory of perception in \emph{De quantitate animae}. A close examination of the text shows, instead, that he is committed to its rejection. I end with some remarks about what it takes for an account of perception to be an extramission theory.
\end{abstract}

% Layout Settings
\setlength{\parindent}{1em}

% Main Content

\section{Augustine and Extramission} % (fold)
\label{sec:augustine_and_extramission}

Augustine is commonly interpreted as endorsing an extramission theory of perception in \emph{De quantitate animae}. 

% section augustine_and_extramission (end)

\section{The definition} % (fold)
\label{sec:the_definition}

When asked by Augustine what sense perception is, Evodius, good Socratic stooge that he is, merely responds with a list, the Peripatetic five senses---vision, audition, olfaction, taste, and touch. Augustine explains that they are presently seeking a single definition that would encompass all five senses (compare Plato, \emph{Theaetetus} 145e--147c) and proposes a candidate definition for Evodius to defend or reject. This definition will subsequently be refined, though it will retain its basic form. It is worth, however, discussing the initial formulation of the definition: 
\begin{quote}
	sensum puto esse non latere animam quod patitur corpus (Augustine, \emph{De quantitate animae}, 23 41)
\end{quote}
Augustine's provisional definition comes in two parts: \emph{quod patitur corpus} describes the object of perception whereas \emph{non latere animam} describes the soul's relation to that object. 

Consider the soul's relation to the object of perception first. And bracket, for the moment, the significance of Augustine's indirect description of that relation. In describing the object of perception as being not hidden from the soul, Augustine is defining perception, in the first instance, as a mode of awareness \citep[275]{Brittain:2002hl}. 

What is the significance of Augustine's indirect description of the soul's relation to the object of perception? Why describe the object of sensory awareness as something that is not hidden from the soul? Some have found this worrying. \citet[112]{Bourke:1947jk} flatly pronounces Augustine's definition ``unsatisfactory'' on just this basis: ``The persistent use of the negative formula (\emph{non latere}) indicates Augustine's inability to say positively just what sensation is.'' And Bourke's judgment is reaffirmed by \citet[104, n. 1]{McMahon:1947dn}. However, I wonder to what extent Bourke's judgment is fair. If we accept, as seems evident, that Augustine had in mind a mode of awareness, then perhaps to describe the object of perception as not hidden from the soul is, after all, to provide a positive characterization of the sensory awareness afforded by perceptual experience.  

Etienne Gilson thought so:
\begin{quote}
	His purpose is to make sensation an activity of the soul within the soul itself. This is really the reason why he defines it in such a roundabout way. The phrase ``\emph{non latet}'' indicates precisely that the soul is a spiritual force, ever watchful and attentive. In order to sense, it does not have to receive anything from the organ that it vivifies; it is enough if the changes undergone by the organs do not escape its notice, and come within the range of its attention. \citep[63]{Gilson:1961ec}
\end{quote}
The positive characterization that Gilson sees in Augustine's roundabout expression is metaphysically significant. We shall return to its significance in due course. But for now, notice how Gilson understands \emph{non latere animam} as the soul's attentive vigilance. 

I believe that Gilson was right to see in Augustine's roundabout expression a positive characterization of sensory awareness. Consistent with Gilson's suggestion, though logically independent of it, one might also understand Augustine as claiming that sensory awareness is a mode of disclosure. To describe the object of perception as not hidden from the soul is to understand sensory awareness not only as the soul's attentive vigilance but also as a mode of disclosure---an activity whereby what was previously hidden from the range of its attention is now revealed to the soul.

Consider now Augustine's description of what is not hidden from the soul, the object of sensory awareness. \citet[274--278]{Brittain:2002hl} observes that there is a crucial ambiguity in \emph{quod patitur corpus}. The verb \emph{patior} means to suffer or undergo, to be affected. So a natural understanding of this phrase might be what the body undergoes. So understood, the object of sensory awareness is a bodily affection, the way in which the body is affected. However, \emph{quod patitur} might also be read as Latin rendering of the Greek {\sbl ὅ τι πάσχει}. So understood, the object of sensory awareness is less a bodily affection than what affects the body. On the former understanding, the object of sensory awareness is internal---it is the perceiver's body being affected in a certain way. On the latter understanding, the object of sensory awareness is external, at least if we rule out cases of auto-affection---it is what affects the body from without. 

\citet[]{Brittain:2002hl} observes that understanding \emph{quod patitur corpus} as what affects the body better coheres with Augustine's own examples of the objects of perception in \emph{De quantitate animae}. The objects of vision, for example, are external bodies located at a distance from the perceiver. Thus Augustine sees Evodius, and Evodius sees Augustine, and neither is at the place where the other is. The objects of perception are external to the perceiver's body. In modern parlane, perception is exteroceptive. And this would remain true even if the sensory disclosure of external bodies involves the perceiver's sensory organs being affected, or even the formation of incorporeal images in the soul occasioned by such affections. 

Understanding \emph{quod patitur corpus} as bodily affection, \emph{passio corporis}, yields a definition of what is, at best, bodily sensation. It at best characterizes a form of interoceptive awareness. It is only on the understanding of \emph{quod patitur corpus} as what affects the body that it plausibly yields a definition of perception. Only so understood does the definition characterize a form of exteroceptive awareness. After all, what affects the body, the objects of perception, are external that body, and the objects of visual perception are located at a distance from the perceiver's body.

\citet[]{ODaly:1987fq} suggests that the ambiguity is intentional and that both sensation and perception involve sensory awareness:
\begin{quote}
	Augustine may wish to distinguish here between sensation (\emph{passio corporis}) and perception (\emph{non latere animam}), but it is probably truer to say that he is rather thinking of the external source of most sensory stimuli when he speaks of \emph{passio corporis}, and including under awareness (\emph{non latere}) all forms of sensation and perception.
\end{quote}

If perception fundamentally involves the soul's sensory awareness understood as a mode of disclosure, and if the objects not hidden from the soul are external bodies, then the emphasis of the expression \emph{quod patitur corpus} is on the fact that the soul's sensory awareness of external bodies is mediated by the body's affection. In this way perception contrasts with the understanding. The soul's awareness of the intelligible is not mediated by the body's affection the way that the soul's sensory awareness is.

% section the_definition (end)

\section{Extramission} % (fold)
\label{sec:extramission}

It is worth getting clear about what, exactly, it means to describe perception as extramissive.

Perception, so conceived, must be, or at least centrally involve, the activity of the perceiver.

This activity is outer directed. In the case of vision, this outer-directed activity is rectilinear.

This outer-directed activity of the perceiver constitutes, at least in part, their perception.

The is outer-directed activity that constitutes, at least in part, the perception of an object involves something spatially extending to the distal object of perception so that at least part of the perceiver is substantially located where the the perceived object is or is at the very least contiguous with it.



% section extramission (end)



% no cite
\nocite{Tourscher:1933rw}

%Bibliography
\bibliographystyle{plainnat}
\bibliography{Philosophy}

\end{document}