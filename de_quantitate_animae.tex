%!TEX TS-program = xelatex 
%!TEX TS-options = -synctex=1 -output-driver="xdvipdfmx -q -E"
%!TEX encoding = UTF-8 Unicode
%
%  de_quantitate_animae
%
%  Created by Mark Eli Kalderon on 2016-01-09.
%  Copyright (c) 2016. All rights reserved.
%

\documentclass[12pt]{article} 

% Definitions
% \newcommand{\change}{\textcolor{blue}{\textbf{CHANGE SLIDE}}}
\newcommand\mykeywords{Augustine, vision, perception, extramission}
\newcommand\myauthor{Mark Eli Kalderon}

% Packages
\usepackage{geometry} \geometry{a4paper} 
\usepackage{url}
% \usepackage{txfonts}
\usepackage{color}
\usepackage{enumerate}
\definecolor{gray}{rgb}{0.459,0.438,0.471}
\usepackage{setspace}
% \doublespace % Uncomment for doublespacing if necessary
% \usepackage{epigraph} % optional

% XeTeX
\usepackage[cm-default]{fontspec}
\usepackage{xltxtra,xunicode}
\defaultfontfeatures{Scale=MatchLowercase,Mapping=tex-text}
\setmainfont{Hoefler Text}
\newfontfamily{\sbl}{SBL Greek}

% Bibliography
\usepackage[round]{natbib}

% Title Information
\title{Perception and Extramission in \emph{De quantitate animae}}
% \author{\myauthor}
\date{} % Leave blank for no date, comment out for most recent date

% PDF Stuff
\usepackage[plainpages=false, pdfpagelabels, bookmarksnumbered, backref, pdftitle={Perception and Extramission in De quantitate animae}, pdfkeywords={\mykeywords}, xetex, colorlinks=true, citecolor=gray, linkcolor=gray, urlcolor=gray, unicode=true]{hyperref} 

%%% BEGIN DOCUMENT
\begin{document}

% Title Page
\maketitle
\begin{abstract}
	\noindent Augustine is commonly interpreted as endorsing an extramission theory of perception in \emph{De quantitate animae}. A close examination of the text shows, instead, that he is committed to its rejection. I end with some remarks about what it takes for an account of perception to be an extramission theory and with a review of the strength of evidence for attributing the extramission theory to Augustine on the basis of his other works.
\end{abstract}

% Layout Settings
\setlength{\parindent}{1em}

% Main Content

\section{Augustine and Extramission} % (fold)
\label{sec:augustine_and_extramission}

Augustine is commonly interpreted as endorsing an extramission theory of perception. Extramissive elements can be found in a number of his works (\emph{De musica} 6.8.21, \emph{De Genesi ad litteram libri duodecim} 1.16, \emph{Sermon} 277.10, \emph{De Trinitate} 9.3). However, at least in his early work, \emph{De quantitate animae}, far from endorsing an extramission theory of perception, Augustine explicitly argues for its rejection. And yet on a standard interpretation, Augustine is understood to endorse the extramission theory in \emph{De quantitate animae} (see, for example, \citealt[82--3]{ODaly:1987fq}). It is perhaps worth considering Augustine's anti-extramission argument in detail. The present essay ends with two sets of reflections. First, we shall provide a diagnosis for the misattribution in terms of unclarity about the commitments of the extramission theory. Specifically, we shall consider what it means to describe an account of perception as extramissive by distinguishing different grades of extramissive involvement, some, if not all, are at the core of Augustine's thinking. Second, we shall briefly review the strength of the evidence for attributing an extramission theory to Augustine on the basis of his other works. We shall see that, at best, it is neither required by scripture nor reason, but represents authoritative opinon, such as Plato's, Ptolemy's, or Galen's, and so is, by Augustine's lights, a defeasible commitment.

% section augustine_and_extramission (end)

\section{The Textual Evidence} % (fold)
\label{sec:the_textual_evidence}

The textual evidence for attributing to Augustine an extramission theory occurs in chapter 23 of \emph{De quantitate animae}. The primary evidence consists in two passages from \emph{De quantitate animae} 23.43, but there is also a back reference to the second passage at the beginning of \emph{De quantitate animae} 23.44.

The first passage is as follows:
\begin{quote}
	is enim se foras porrigit, et per oculos emicat longinus quaquaversum potest lustrare quod cernimus. 
	% Unde fit ut ibi potius vieat ubi est id quod videt, non unde erumpit ut videat.
	
	Sight extends itself outward and through the eyes dart forth in every possible direction to light up what we see. (\emph{De quantitate animae} 23.43, \citealt[66]{Colleran:1949ys})
	% Hence it happens that it sees rather in the place  where the object seen is present, not in the place from which it goes out to see. 
\end{quote}
As Colleran's translation indicates, the Latin pronoun \emph{is} takes as its antecedent \emph{visus} from the previous line. \emph{Visus} can be translated as vision or sight, but sight is the appropriate translation as the present passage seems to be describing its actualization. On the extramissionist reading, this passage is describing the emission of the visual ray.

The second passage involves the stick analogy that Alexander of Aphrodisias attributes to the Stoics in \emph{De anima} 130.14:
\begin{quote}
	Visu, inquam, porrecto in eum locum in quo es, video te ubi es: at me ibi non esse confiteor. Sed, quemadmodum si virga te tangerem, ego utique tangerem, idque sentirem, neque tamen ego ibi essem ubi te tangerem. 
	
	I say that be means of sight, reaching out to that place where you are, I see you where you are. But that I am not there, I admit. Still let us suppose that I were to touch you with a stick: I certainly would be the one doing the touching and I would sense it; yet I would not be there where I touched you. (\emph{De quantitate animae} 23.43, \citealt[66]{Colleran:1949ys}) 
\end{quote}
On the extramissionist reading, seeing by means of the the visual ray is likened to touching something with a stick. The visual ray, like the stick, is a rectilinear, continuous unity.

One issue concerning either passage is their dialectical context. Both occur in chapter 23 as part of an extended discussion of perception that only culminates in chapter 30. In chapter 23, Augustine presents Evodius with an argument about seeing at a distance the full force of which he is only able to appreciate in chapter 30 once certain conceptual obstacles are removed. These include getting clear on the difference between perception and knowledge, the difference between reasoning and reason, and understanding the rules for constructing and evaluating definitions. Claims made at this stage of the dialogue are unlikely to be definitive. We shall have more to say about this as we proceed.

Let's begin with the first passage. It is not an unambiguous statement of the extramission theory. First, that sight extends itself outwards through the eyes seems like a reasonable description of looking and seeing, at least as a Platonist conceives of it. But the outward activity of looking and seeing is not the exclusive provenance of the extramission theory. The Platonic element of Augustine's description consists in two things, the first more explicit then the next.  First, sight is a power of the soul that is exercised through the use of the eyes. It is an instrument of the soul (\emph{De quantitate animae} 33.41, \emph{De Genesi ad litteram libri duodecim} 12.24.51). That the body is an instrument of the soul is a distinctively Platonic thought (compare Plotinus, \emph{Ennead} 4.7.8). Second, Augustine is here emphasizing how seeing is an activity of sight, a power of the soul. The superiority of the incorporeal soul is manifest in its ability to act upon the sensible and corporeal without the sensible and the corporeal being able, in turn, to act upon the soul (\emph{De musica} 6.5.8--10; see \citealt{Silva:2014bh} for discussion). So it is the soul that acts in seeing and so places itself in the distal body seen (\citealt[205, n.55]{Colleran:1949ys}). The emphasis here is on the activity of the soul as opposed to its being substantially located where the perceived object is.

The outward activity of looking and seeing is not the exclusive provenance of the extramission theory. One might object that this ignores the illuminationist imagery at the end of the passage---sight darts forth from the eyes and illuminates what it sees. While it is true that the visual ray of the extramission theory is often likened to light (Empedocles, \textsc{dk} 31b84, Plato, \emph{Timaeus} 45b−c), the illuminationist imagery, considered by itself, is insufficient grounds for the attribution of the extramission theory to Augustine. Other thinkers, who have explicitly rejected the extramission theory, have coherently embraced this imagery. The illuminationist imagery is undoubtedly of Neoplatonic origin, but neither Plotinus nor Porphyry are extramission theorists. Neither are Iamblichus, Proclus, Priscian, nor Pseudo-Simplicius. In addition, later scholastic thinkers influenced by Augustine employ the Neoplatonic illuminationist imagery while disavowing the extramission theory. Thus Peter John Olivi compares a perceiver's gaze to light illuminating its object (\emph{Questiones in secundum librum Sententiarum} q. 72 35–36) even as he denies that seeing involves any real emission (\emph{Questiones in secundum librum Sententiarum} q. 58 ad 14.8). 
% \citet[185]{Merleau-Ponty:1967fj} will echo both aspects of Olivi's position. For Merleau-Ponty, the illuminationist imagery captures the active outward phenomenology of looking and seeing even if the extramission theory provides a false causal model of perception.

How might the illuminationist imagery be understood if it is not, indeed, committed to the extramission theory? The awareness afforded by visual experience is like a beam of light that manifests the latent presence of its object. Vision, like illumination, has direction. Light is emitted outward from its source upon the scene that it illuminates. Vision too is outer-directed. In seeing, the perceiver looks out upon the scene before them. Not only do vision and light have direction but they are both rectilinear as well. Moreover, just as illumination manifests the latent visibility of an object, seeing an illuminated object manifests its latent presence to the perceiver revealing it to be where it is. The explicit awareness of the natural environment afforded by visual experience is akin to light not only in its rectilinear directionality and its power to manifest latent presence, but in the manner in which it discloses distal aspects of that environment. Just as the illumination alights upon the object it illuminates at a distance from its source, the perceiver’s gaze alights upon the object of perception at a distance from their eyes. The imagery here not only emphasizes that vision is a kind of perception at a distance but invokes an active outward extension. The acceptance of the analogy, so explicated, is consistent with no part of the perceiver being substantially located where the perceived object is and so bears no commitment to the active outward extension being specifically spatial. 

% a beam of light may “pose” on an object that it illuminates and that it reaches from a distance, the perceiver’s gaze may “pose” on the object that it presents and that it reaches from a distance. T

% In addition to the phenomenological aptness of the illuminationist imagery, Augustine may have had another motive in deploying it, one that is consistent with the rejection of the extramission theory. Corporeal light may be the means by which we see, but spiritual light is the means by which we understand. The doctrine of illumination was first stated in \emph{De Magistro} written during the same period as the present dialogue (on the doctrine of illumination see \citealt{Allers:1952os} and \citealt{Matthews:2014aa}). Augustine understands corporeal light as the image, in the Platonic sense, of the true spiritual light.  Insofar as perception, like the understanding, involves a mode of awareness, the illuminationist imagery may be the joint result of Augustine's doctrine of illumination---that spiritual light is that by which we understand---and the claim that perceptual activity is an image of intellectual activity. Perception, like the understanding, is a mode of awareness afforded by illumination, albeit of a corporeal variety.

% Finally, a defender of the extramissionist reading might object that we are overlooking the final line of the quoted passage where it is maintained that vision occurs where the object is seen and not at the place where sight goes out from. It is true that if vision occurs through the emission of a visual ray, then this is well explained. However, while the extramission theory may explain this claim, that explanation is not indispensable. Augustine, here, is echoing a Neoplatonic doctrine, that perception places the perceiver in the object perceived, shared by Plotinus and Porphyry, neither of whom were extramissionists. Augustine seems to have in mind, here, not the extramission theory, but the phenomenology of attentive awareness. Perception places us in the perceived body. That is where the perceiver’s attentive awareness is. In selectively attending to an object of your experience, where is your attention? On the object selectively attended to, of course. The query, ``Where’s your head at?'', gives expression to this. Your ``head'' is where the object of your attention is. In general, an act or episode of attention is where its object is. Attention is not the kind of thing that has location in itself. Attention is conceived by Augustine to be an incorporeal power of the soul to direct the mind to various objects. Insofar as attention can be said to have location, it must inherit this location from the location of its object. This, then, is the claim about the phenomenology of attentive awareness, that attention is located where its object is. One need not be an extramission theorist to endorse it.

The first passage, while not, by itself, sufficient grounds for attributing to Augustine an extramission theory, may gain new force and significance, however, when considered in context with the second passage. Recall, there, Evodius likens looking and seeing a distal object to touching it with a stick. On the extramissionist reading, this is a description of visual rays extending from the perceiver to the distal object. Like the stick, the visual ray is a continuous unity that spatially extends from the perceiver to the object perceived. Should we accept the extramissionist reading? A number of observations are relevant here.

First, Evodius is adapting the Stoic stick analogy (Alexander of Aphrodisias, \emph{De anima} 130 14).  The Stoic analogy is criticized by Galen in \emph{De Placitis Hippocratis et Platonis} 7.7 and echoed by Descartes in \emph{La Dioptrique}. That analogy is plausibly rooted in, and an interpretation of, Plato's \emph{Timaeus} 45b−c discussion of perception (see \citealt[chapter one]{Lindberg:1977aa}). A primeval fire within the eye, that gives light but does not burn, extends out through the pupil. There it encounters what is like it, external light. And together they constitute a continuous unity, a compound of emitted light and external light, extending from the perceiver to the distant object of perception in a rectilinear path. The stick analogy captures the formal features of the \emph{Timaeus} account, namely, that the compound of emitted light and external light constitutes a continuous, rectilinear unity---just like a straight stick. However, these formal features are preserved in accounts given by thinkers that explicitly reject the extramission theory. And if that is right, then the stick analogy is not sufficient for the attribution of an extramission theory. 

Consider, then, one prominent account. Aristotle rejects the extramission theory as providing a false causal model of perception (\emph{De sensu} 2 438\( ^{a} \)26--438\( ^{b} \)2), and yet his alternative causal model preserves these formal features. The illuminated media intervening between the perceiver and the distant object of perception already, according to Aristotle, constitutes a continuous unity (\emph{De anima} 2.7 419\( ^{a} \)12--22). But the resulting account is broadly intromissionist in that it emphasizes the perceived object acting upon the perceiver, albeit mediately. Specifically, the perceived object immediately acts upon the medium, a continuous unity, which in turn immediately acts upon the perceiver. In this way, the perceived object mediately acts upon the perceiver. 

Acceptance of the Stoic stick analogy is not sufficient for commitment to the extramission theory. Indeed, Descartes, in \emph{La Dioptrique}, accepts the Stoic analogy even though his mechanical explanation of vision is in no way extramissive. Nor, indeed, is it necessary.  Galen, in \emph{De Placitis Hippocratis et Platonis} 7, rejects the Stoic analogy even as he embraces the extramission theory.

Another worry concerns the specific use to which Evodius is putting the Stoic stick analogy. It is commonly accepted that the extramission theory is one way of modelling vision on the basis of touch. Touch may be a contact sense, but distal senses, such as vision, may be modelled on touch if the emitted visual ray extends to the distant object of perception and is in contact with it. A specific form of touch is here accepted as paradigmatic, namely, sensation by contact. But sensation by contact is not the only form of touch. Thus, \citet{Broad:1952kx} distinguishes a dynamical form of touch---haptic touch in modern parlance---from sensation by contact. But the conception of touch involved in Evodius' adaption of the Stoic stick analogy is not sensation by contact. Rather, it is a specific and atypical form of haptic touch. It is a form of haptic touch since touching something with a stick---``poking''---is an activity and so displays the dynamical character of haptic touch. What is distinctive about this form of haptic touch is that involves a form of distal touch, where the felt object is not in direct contact with the perceiver's body. This form of haptic touch is atypical in that most forms of haptic touch involve contact and not all involve an instrument or intervening medium. So, consider feeling the wooden frame through the padding of a Victorian hobby horse. The perceiver feels the wooden frame even though they are not in direct contact with it. Instruments, such as a stick, can be exploited in distal touch. Thus by means of a stick one may feel the texture of a distant surface, or its hardness and rigidity. One feels the tactile qualities of the distant surface through the stick, despite not being in direct contact with it. 

The advertised worry turns on two observations. First, Evodius' appeal to distal touch is in tension, if not inconsistent, with the extramission theory. The extramission theory is motivated by the idea that the perceived object must be in contact with the sense. But, again, the paradigmatic conception of touch is sensation by contact and not distal touch. In a way, Augustine, at this point of the dialogue, has elicited a dialectical concession from Evodius (albeit one whose full significance Evodius will only realize in chapter 30). The extended discussion of perception is inaugurated by a puzzle Evoidus' raises about the incorporeal nature of the soul. If the soul is not extended throughout the body, then how can it feel wherever the body is touched? Evodius' implicit thought, here---that the perceived object must be in contact with the sense, understood not merely as a the sense organ, but that organ as animated by the sensitive soul, the principle of sensation---is now abandoned. One can feel what one is not in direct contact with. It is because Evodius is walking back from a previous commitment, shared with the extramission theory, that this is properly regarded as a dialectical concession.

Second, there is an oddity in Evodius' description of distal touch. Using an instrument, such as a stick, to feel a distant textured surface, it seems, from within, that we experience that texture at the end of the stick. ``As if your nervous system had a sensor out at the tip of the wand'', as \citet[47]{Dennett:1993ce} observes. That is, in cases of distal touch, you feel tactile qualities where they are. But Evodius seems to disavow that: ``I certainly would be the one doing the touching and I would sense it; yet I would not be there where I touched you'' (\emph{De quantitate animae} 23.43). But such a disavowal seems to be in tension with, not only the phenomenology of distal touch, but also with the extramission theory as usually understood. If the eye emits a visual ray that extends to the distant object of perception so that it is in contact with it, then at least a part of the perceiver is substantially located where the perceived object is, or is, at the very least, contiguous with it. And it is there, at the point of contact, that the object is sensed. Compare the view that Nemesius attributes to Hipparchus, ``Hipparchus says that rays extend from the eyes and with their extremities lay hold on external bodies like the touch of hands'' (\emph{De natura hominis} 7; \citealt[104]{Sharples:2008aa}).
 
% section the_textual_evidence (end)

\section{Dialectical Context} % (fold)
\label{sec:dialectical_context}

The force of the textual evidence for attributing an extramission theory to Augustine, in \emph{De quantitate animae}, crucially depends upon the dialectical context. Allow me to briefly review the place of Augustine's account of perception in that dialogue.

\emph{De quantitate animae}  is mainly charged with the task of arguing for the incorporeal nature of the living soul. In the dialogue, Evodius, like Augustine’s former self (\emph{Confessions} 7.1ff), has a hard time conceiving of something that is both real and incorporeal (compare the position of the Giants in Plato's \emph{Sophist}). Throughout \emph{De quantitate animae}, Augustine will give accounts of the soul’s powers and activities that are meant to persuade us that these are not powers and activities of the body. The question of the soul’s incorporeal nature is linked with the question of its magnitude. Bodies are extended in three dimensions. If souls are inextended, if they lack extensive magnitude, then they are incorporeal. But, importantly, being incorporeal is consistent with the soul’s possession of superior virtual magnitude. That is to say that psychic powers, the powers and virtues of the soul, are superior to any corporeal power. 

The question concerning the magnitude of the soul is subject to further specification since two senses of magnitude may be distinguished (\emph{De quantitate animae} 3.4):
\begin{enumerate}
	\item Extensive magnitudes: magnitudes of extension, “How tall is Hercules?”
	\item Virtual magnitudes: magnitudes of power, “How great is Hercules valour or prowess?”
\end{enumerate}
Evodius seeks an answer to the question in both senses. In the sense of extensive magnitude, Augustine denies that the soul has quantity at all. The soul is inextended, and, hence, incorporeal since corporeal bodies are necessarily extended in three dimensions. Augustine's denial concerns continuous quantities like magnitude and not discrete quantities like number. In denying that the soul has quantity, Augustine is not denying that there are numerically distinct souls. Augustine will maintain that the soul, while lacking extensive magnitude, nevertheless possesses virtual magnitude. Whereas the question how great is the soul in the sense of extensive magnitudes is answered in the negative in \emph{De quantitate animae} 3.4, a full answer to the question how great is the soul in the sense of virtual magnitude, in its powers and virtues, only emerges in the hierarchically organised enumeration of the soul’s powers that ends the dialogue (\emph{De quantitate animae} 33–36; see \citealt{Brittain2003-BRICA-2} for discussion). This hierarchically organised enumeration of the soul’s powers is also, at the same time, a soteriology, at least in part, in that it describes the soul’s ascent to God (and it is in this sense that it is a theoretical articulation of the vision in Ostia that Augustine shared with Monica as reported in the \emph{Confessions}).

Evodius will resist Augustine’s denial that the soul possesses extensive magnitude. Extensive magnitudes cited by Augustine are length, width, and strength. A third spatial dimension, height, is latter added, \emph{De quantitate animae} 6. Strength here translates \emph{robustam} which means the resistance offered by solid objects that occupy space. Specifically, then, Evodius doubts whether something without length, width, height, and strength so much as could exist. 
% This occasions puzzlement in Evodius. The source of his puzzlement is parochial and linguistic. Latin lacked unambiguous terms pick out the three dimensions, and so Evodius does not immediately pick up on Augustine’s intention in speaking of \emph{altitudo}. He has the concept of the third dimension, he merely lacks a Latin word that will immediately and unambiguously pick it out.
Here, Evodius is echoing the position of the Giants. In the \emph{Sophist}, Plato re-envisions the Gigantomachy, the struggle for political supremacy over the cosmos between the Giants and the Olympian Gods, as a metaphysical dispute between corporealists, the Giants, and the Friends of the Forms, the Gods. Compare Evodius position to the Giants:
\begin{quote}
	One party is trying to drag everything down to earth out of heaven and the unseen, literally grasping rocks and trees in their hands, for they lay hold upon every stock and stone and strenuously affirm that real existence belongs only to that which can be handled and offers resistance to the touch. (Plato, \emph{Sophist} 246a; Cornford in \citealt[990]{Hamilton:1989fk})
\end{quote}
In response, Augustine will offer a negative argument and a positive argument. According the negative argument, just because the soul lacks extension does not mean that it is not real (\emph{De quantitate animae} 3.4–4.5). And according to the positive argument, the soul must be incorporeal since it possesses powers that bodies lack (\emph{De quantitate animae} 4.6–15.25).

In the \emph{Sophist}, the Eleatic Visitor convinces the Giants to modify their corporealism in order to allow for justice, since denying the existence of this virtue would be impious. Justice lacks length, width, and height. It cannot be grasped and offers no resistance to touch and hence lacks strength, \emph{robustam}. And it is by means of the Eleatic Visitor’s argument that Augustine convinces Evodius that lacking extensive magnitude does not entail nonexistence. Specifically, a tree, a sensible and corporeal object with extensive magnitude, exists. But so does justice despite lacking extensive magnitude. Moreover, and importantly, justice is greater in value than the tree. The adaption of the Eleatic Visitor’s argument is meant to establish not only that virtues like justice may exist despite being inextended but that they may also be more excellent than any extended thing. Justice may lack extensive magnitude and yet possess greater virtual magnitude than a sensible body. 

Augustine may have established that the soul’s nonexistence does not follow from its lack of extension, but he has yet to establish what the soul positively is. Moreover, just because justice is real despite being inextended, it does not follow that the soul itself is inextended. Evodius is persistently attracted to the idea that the soul extends throughout the body that it animates and hence must itself be extended. Augustine will argue, in contrast, that the soul is inextended because it possesses powers that corporeal extended things lack. In effect, Augustine is arguing that the soul possesses greater virtual magnitude than any extended corporeal thing. It is the greatness of the soul that establishes its inextended, and hence, incorporeal nature.

Augustine’s argumentative strategy is to emphasise how an account of the powers and activities of the soul establish that they are not powers and activities of a body. Indeed, so construed, Augustine offers not one positive argument but two:
\begin{enumerate}
	\item In visually imagining a remembered body, my mental image is not constrained in the way that corporeal images are. Corporeal likenesses are only as large as the body in which the image occurs, incorporeal likenesses are not so constrained (\emph{De quantitate animae} 5.7–9)
	\item In conceiving of geometrical figures abstracted from three-dimensional bodies—such as planes, lines, and points that figure as parts of those bodies—the soul must be incorporeal since only like can conceive of like (\emph{De quantitate animae} 6.10–15.26)
\end{enumerate}
At this point the central case of the dialogue has been made. That is to say, Augustine has argued that just because something is incorporeal does not mean that it is less real or less valuable than something corporeal. Moreover, Augustine has argued that the soul must be incorporeal because it possesses powers that corporeal extended things lack. And that the incorporeal powers of the soul are either more valuable than their corporeal counterparts or at least more valuable than the corporeal objects of their activity. 

% More specifically, Augustine has completed his negative and positive arguments. 

Evodius accepts that Augustine has established this. What are raised, at this point, are less objections \emph{per se} to Augustine's conclusions than certain residual puzzles or \emph{aporiai}. There are two:
\begin{enumerate}
	\item If the soul is inextended, how is it that the soul grows over time as the body grows? (\emph{De quantitate animae} 15.26–22)
	\item If the soul is inextended, then it is not extended throughout the body. But if the soul is not extended throughout the body, then how can it feel wherever the body is touched? (\emph{De quantitate animae} 23–30)
\end{enumerate}
It is the second residual puzzle that prompts the extended discussion of perception.
	
A tacit assumption is at work in Evodius' second puzzle. Attending only to sensation by contact, and perhaps by regarding it as an exemplary form of perception, can suggest that the principle governing sensation takes a certain a form. Its slogan might be: to be perceptible is to be palpable (see \citealt{Kalderon:2015fr} for discussion). The idea is that the perceived object must be in contact with the principle governing sensation. The sensitive soul is the principle governing sensation. Wherever Evodius is touched upon his body he feels it. That means that the principle of his sensation, Evoidus' sensitive soul, must be in contact with what touches his body, if it is to be perceptible. But it is perceptible. And so the sensitive soul, at least, must be extended throughout the body.

Augustine is sensitive to this tacit assumption even though he does not make it explicit. This sensitivity is manifest in a curious shift in example. Whereas Evodius' objection turns on an observation about touch, understood as sensation by contact, Augustine develops his definition of perception with reference to vision, a form of distal perception. The objects of vision must be at a distance from the perceiver's eyes. Augustine's intent is to undermine the tacit assumption driving Evodius to think that the soul must be extended throughout the body if sensation by contact is to be so much as possible. Augustine's intent is to undermine the assumption that to be perceptible is to be palpable, that the perceived object must be in contact with the principle governing sensation. This assumption not only drives Evodius' puzzle, but the extramission theory as well. 

% section dialectical_context (end)

\section{The Definition} % (fold)
\label{sec:the_definition}

When asked by Augustine what sense perception is, Evodius, good Socratic stooge that he is, merely responds with a list, the Peripatetic five senses---vision, audition, olfaction, taste, and touch. Augustine explains that they are presently seeking a single definition that would encompass all five senses (compare, for example, Plato, \emph{Theaetetus} 145e--147c) and proposes a candidate definition for Evodius to defend or reject. This definition will subsequently be refined, though it will retain its basic form. It is worth, however, discussing the initial formulation of the definition: 
\begin{quote}
	sensum puto esse non latere animam quod patitur corpus (Augustine, \emph{De quantitate animae}, 23.41)
\end{quote}
Augustine's provisional definition comes in two parts: \emph{quod patitur corpus} describes the object of perception whereas \emph{non latere animam} describes the soul's relation to that object. 

Consider the soul's relation to the object of perception first. And bracket, for the moment, the significance of Augustine's indirect description of that relation. In describing the object of perception as being not hidden from the soul, Augustine is defining perception, in the first instance, as a mode of awareness \citep[275]{Brittain:2002hl}. 

What is the significance of Augustine's indirect description of the soul's relation to the object of perception? Why describe the object of sensory awareness as something that is not hidden from the soul? Some have found this worrying. \citet[112]{Bourke:1947jk} flatly pronounces Augustine's definition ``unsatisfactory'' on just this basis: ``The persistent use of the negative formula (\emph{non latere}) indicates Augustine's inability to say positively just what sensation is.'' And Bourke's judgment is reaffirmed by \citet[104, n.1]{McMahon:1947dn}. However, I wonder to what extent Bourke's judgment is fair. If we accept, as seems evident, that Augustine had in mind a mode of awareness, then perhaps to describe the object of perception as not hidden from the soul is, after all, to provide a positive characterization of the sensory awareness afforded by perceptual experience.

Etienne Gilson thought so:
\begin{quote}
	His purpose is to make sensation an activity of the soul within the soul itself. This is really the reason why he defines it in such a roundabout way. The phrase ``\emph{non latet}'' indicates precisely that the soul is a spiritual force, ever watchful and attentive. In order to sense, it does not have to receive anything from the organ that it vivifies; it is enough if the changes undergone by the organs do not escape its notice, and come within the range of its attention. \citep[63]{Gilson:1961ec}
\end{quote}
Notice how Gilson understands something not being hidden from the soul as the effect of the soul's attentive vigilance. The positive characterization that Gilson sees in Augustine's roundabout expression is metaphysically significant. That something is not hidden from the range of the soul's attention is due to the activity of the soul within the soul itself. I believe that Gilson was right to see in Augustine's roundabout expression a positive characterization of sensory awareness. Consistent with Gilson's suggestion, one might also understand Augustine as claiming that sensory awareness is a mode of disclosure. To describe the object of perception as not hidden from the soul is to understand sensory awareness not only as the soul's attentive vigilance but also as a mode of disclosure---an activity whereby what was previously hidden from the range of its attention is now revealed to the soul.

Consider now Augustine's description of what is not hidden from the soul, the object of sensory awareness. \citet[274--278]{Brittain:2002hl} observes that there is a crucial ambiguity in \emph{quod patitur corpus}. The verb \emph{patior} means to suffer or undergo, to be affected. So a natural understanding of this phrase might be what the body undergoes. So understood, the object of sensory awareness is a bodily affection, the way in which the body is affected. However, \emph{quod patitur} might also be read as Latin rendering of the Greek {\sbl ὅ τι πάσχει}. So understood, the object of sensory awareness is less a bodily affection than what affects the body. On the former understanding, the object of sensory awareness is internal---it is the perceiver's body being affected in a certain way. On the latter understanding, the object of sensory awareness is external, at least if we rule out cases of auto-affection---it is what affects the body from without. 

\citet[]{Brittain:2002hl} observes that understanding \emph{quod patitur corpus} as what affects the body better coheres with Augustine's own examples of the objects of perception in \emph{De quantitate animae}. The objects of vision, for example, are external bodies located at a distance from the perceiver. Thus Augustine sees Evodius, and Evodius sees Augustine, and neither is at the place where the other is.
% The objects of perception are external to the perceiver's body. In modern parlance, perception is exteroceptive. And this would remain true even if the sensory disclosure of external bodies involves the perceiver's sensory organs being affected, or even the formation of incorporeal images in the soul occasioned by such affections.
Understanding \emph{quod patitur corpus} as bodily affection, \emph{passio corporis}, yields a definition of what is, at best, bodily sensation. It at best characterizes a form of interoceptive awareness. It is only on the understanding of \emph{quod patitur corpus} as what affects the body that it plausibly yields a definition of perception. Only so understood does the definition characterize a form of exteroceptive awareness. After all, what affects the body, the objects of perception, are external that body, and the objects of visual perception are located at a distance from the perceiver's body.  Moreover, as we have emphasized (section~\ref{sec:dialectical_context}), it is dialectically important that Augustine is discussing vision, a distal sense.

% \citet[]{ODaly:1987fq} suggests that the ambiguity is intentional and that both sensation and perception involve sensory awareness:
% \begin{quote}
% 	Augustine may wish to distinguish here between sensation (\emph{passio corporis}) and perception (\emph{non latere animam}), but it is probably truer to say that he is rather thinking of the external source of most sensory stimuli when he speaks of \emph{passio corporis}, and including under awareness (\emph{non latere}) all forms of sensation and perception.
% \end{quote}

If perception fundamentally involves the soul's sensory awareness understood as a mode of disclosure, and if the objects not hidden from the soul are external bodies, then the emphasis of the expression \emph{quod patitur corpus} is on the fact that the soul's sensory awareness of external bodies is mediated by the body's affection. In this way perception contrasts with the understanding. The soul's awareness of the intelligible is not mediated by the body's affection the way that the soul's sensory awareness is. This clearly shows that \emph{quod patitur corpus} or \emph{passio corporis} are not schematic placeholders for a broadly Galenic physiology. Rather, the body's affection is used to specify the relevant kind of awareness involved in perception, that is to say, a sensory, as opposed to an epistemic, awareness (\emph{De quantitate animae} 29).

More refined versions of the preliminary definition will be given in \emph{De quantitate animae} 25.48, 26.49, and 30.59.
% :
% \begin{quote}
% 	sensus sit passio corporis per seipsam non latens animam (\emph{De quantitate animae} 25.48)
%
% 	sensus est cum passio corporis per seipsam non latet animam (\emph{De quantitate animae} 26.49)
%
% 	sensus est corporis passio per seipsam non latens animam (\emph{De quantitate animae} 30.59)
% 	\end{quote}
% Perception will be what affects the body by itself not being hidden from the soul
% Perception is when what affects the body by itself is not hid from the soul
% Perception is what affects the body by itself not hid from the soul
There are two salient differences between the preliminary definition and the refined versions.

First, \emph{quod patitur corpus} has been replaced by \emph{passio corporis} (or, equivalently, \emph{corporis passio}—the variation in word order makes no difference to the Latin grammar). \emph{Passio corporis} is most naturally read, not as what affects the body, but the body's affection, what the body undergoes when being affected from without. However, as we observed before, so understood, the definition does not define perception but bodily sensation. But it is clear and important that Augustine is not only discussing perception but a form of distal perception, vision. So \emph{passio corporis} is a kind of Augustinian shorthand for \emph{quod patitur corpus} that is meant to inherit its ambiguities. So this first difference is superficial and does not represent a genuine refinement of the definition.

Second, each of the refined versions include a new phrase, \emph{per seipsam}, which can be translated as by itself or directly. Unlike the substitution of \emph{passio corporis} for \emph{quod patitur corpus}, this is not a superficial difference but does represent a genuine refinement of the definition. How are we to understand this refinement and what is the nature of its grounds?

The refinement of the preliminary definition was prompted by a counterexample to its sufficiency. Growth is an affection of the body not hidden from the soul and yet it is insensible. According to Augustine, such failures, where the definition encompasses more than it undertook to explain, can sometimes be fixed with an emendation (\emph{De quantitate animae} 25.47). Thus, Augustine's challenge to Evodius (\emph{De quantitate animae} 25.48): What qualification can be added to the definition to render it valid? Evodius' answer (picking up on an earlier claim of Augustine's, \emph{De quantitate animae} 24.46, whose significance he did not, at that time, understand) is \emph{per seipsam}. What is the force of the qualification? Growth is an affection of the body not hidden from the soul and yet growth is insensible. That is why it represented a failure of sufficiency of the initial formulation of the definition. However, it is not true that growth \emph{by itself} is not hidden from the soul. Growth only falls within the range of the soul's attention by the operation of reason and intellect. In contrast, the perceived object affecting the eyes is by itself not hidden from the soul.

The structure of Augustine's definition is metaphysically telling. The distal object acts upon the animated sense organ occasioning the sensitive soul to direct its attentive awareness to what affects that organ. First, notice that it is only the sense organ, and not the sensitive soul itself, that is affected by the object of perception. Plausibly, this is an instance of the prohibition on ontological inferiors acting upon ontological superiors. The sensible and the corporeal may act upon the sense organ, but not the sensitive soul that animates it (\emph{De musica} 6.5.8–10; see \citealt{Silva:2014bh} for discussion). Moreover, as we shall see in the next section, the sense organ is only affected in the way that it is because it is animated by the sensitive soul (compare Priscian, \emph{Metaphrasis} 2.1--2). Though the object of perception does not act upon the sensitive soul, it is not hidden from the range of the soul's attention. That something is not hidden from the range of the soul’s attention is due to the activity of the soul within the soul itself.

Not only will Augustine criticise the preliminary definition to motivate its refinement, but he will also clarify a number of issues that might stand in the way of accepting the definition. These tasks will be pursued in chapters 24 to 29. It is only when we have the refined definition, and the obstacles to accepting it have been eliminated, does Augustine apply that definition, in chapter 30, to answer Evodius' challenge to the incorporeality of the soul---to explain how it is that the soul feels wherever the body is touched if it is not extended throughout the body.

% These include getting clear on the difference between perception and knowledge, the difference between reasoning and reason, and understanding the rules for constructing and evaluating definitions. 

% section the_definition (end)

\section{Augustine's Counterargument} % (fold)
\label{sec:augustine_s_counterargument}

Augustine counterargument begins with a restatement of the refined definition whose validity has been accepted by Evodius and Augustine (\emph{De quantitate animae} 25.48, 26.49, and 30.59):
\begin{enumerate}[1.]
	\item Perception is what affects the body by itself not being hidden from the soul.
\end{enumerate}
From the accepted definition it immediately follows that:
\begin{enumerate}[2.]
	\item If one perceives, one's body is acted upon.
\end{enumerate}
Moreover, it was earlier (\emph{De quantitate animae} 23.42) agreed that:
\begin{enumerate}[3.]
	\item Seeing is a form of perception.
\end{enumerate}
But seeing is in some ways a distinctive form of perception. It is a mode of distal perception. Importantly, it was also agreed that:
\begin{enumerate}[4.]
	\item A perceiver sees an object where the perceiver is not.
\end{enumerate}
If sight were confined to where the perceiver is at, at best the eye alone could be seen. But the objects of sight are located at a distance from the perceiver. So the perceiver sees an object where they are not. 

From these premises a startling conclusion follows. Seeing is a form of perception. If one perceives, one is acted upon. So in seeing one is acted upon. And since what acts upon one when one sees is located where one is not, in seeing one is acted upon where one is not. The conclusion involves attributing to the animate eye a passive power to be affected where it is not. This is a passive power not shared with soulless natural bodies. Inanimate natural bodies can only be acted upon by what is in contact with them. The living eye, part of a whole and healthy animal, is animated by the sensitive soul, and it is a manifestation of the superiority of the soul that it endows the eye that it animates with the passive power to be affected where it is not. The sensitive soul may lack extensive magnitude, but it has, in this way, great virtual magnitude. This nicely fits the pattern we observed in Augustine's postive argument for the intextended nature of the soul. This involved attributing superior powers to the soul not possessed by extended corporeal things. Thus while corporeal images are limited by the size of the body upon which they are inscribed, the soul's power to recall an image of an object previously seen is not subject to this limitation. Similarly, the soul possesses the power to conceive of incorporeal geometrical abstractions, and since only like may conceive of like, the soul itself must itself be incorporeal. It thus possesses a power that no corporeal thing may have. The passive power to be affected where one is not conferred by the sensitive soul is similarly a power that no soulless body may possess. Further testimony to the greatness of soul, conceived, not as greatness of extent but greatness of power or virtue.

A commitment to the animate eye possessing the passive power to be affected where it is not is one that Augustine shares with thinkers who explain the eye's affection in terms of sympathy (Cleomedes \emph{Meteōra}, Epicurus, \emph{Letter to Herodotus}, Plotinus \emph{Ennead} 4.4-6, Porphyry \emph{Ad Gaurum}). The sympathetic affection of the animate eye is the result of action at a distance, and so the animate eye possesses a passive power to be affected where it is not. In contrast with the Stoics, Epicureans, and Neoplatonists, Augustine does not explain this passive power in terms of sympathy, nor indeed in terms of any other principle. We are merely meant to marvel at this power that the sensitive soul endows the eye with, a power superior to any passive power of a soulless natural body.

How does this attribution of a passive power possessed by no purely corporeal thing bear on Evodius' puzzle? Recall the tacit assumption behind Evodius' use of sensation by contact, that the object of sensation must be in contact with its principle, that by which one senses. It is just this assumption that drives Evodius conviction that the soul must be extended throughout the body. For only if it were could it be in contact with what touches the body. But if the sensitive soul confers the passive power to be affected where one is not, then there is no need for the sensitive soul to extend throughout the body. The soul need not be where the body is affected, at least substantially, for this affection to be not hidden from the range of its attention (\emph{De quantitate animae} 30.59). 

Though the writing has been on the wall since chapter 23, Evodius is stunned:
\begin{quote}
	That conclusion upsets me very much, so much, in fact, that I am completely stunned. I do not know what to answer and I do not know where I am. What shall I say? Shall I say that a bodily [affection] of which the soul is aware directly is not sensation? What is it, then, if it is not that? Shall I say that the eyes [are affected by] nothing when we see? That is most absurd. Shall I say that the eyes [are affected] where they are? But they do not see themselves and nothing is where they are, except themselves. Shall I say that the soul is not more powerful than the eyes, when the soul is the very power of the eyes? Nothing is more unreasonable. Or must this be said, that it is a sign of greater power to [be affected] there where something is than to [be affected by] it where it is not? But, if that were true, sight would not be rated higher than the other senses. (\emph{De quantitate animae} 30.60; \citealt[87]{Colleran:1949ys}. As we observed earlier, section~\ref{sec:the_definition}, \emph{patior} can be translated as to suffer, to undergo, to be affected. Colleran, here, translates it as to experience. This is a reasonable piece of ordinary usage. Unfortunately, Anglophone philosophy has since been in the grips of an extraordinary usage, see \citealt{Hinton:1973js}, thus rendering Colleran's translation potentially misleading.)
\end{quote}
The reason that corporeal things lack the passive power to be affected where they are not is because, being corporeal and extended, they are confined to the place where they are. The soul, being inextended and not so confined, may confer this passive power on a body that it animates.

Augustine thus explicitly rejects the principle that drove Evodius' \emph{aporia}---to be perceptible is to be palpable to sense, that the perceived object must be in contact with the principle governing sensation. But this is the principle that drives extramission theories as well. Thus Augustine's counterargument is an anti-extramis\-sion argument. A failure to recognize this, and so misattribute to Augustine a commitment to extramission, is due, at least in part, to unclarity about the commitments of the extramission theory.

% section augustine_s_counterargument (end)

\section{The Grades of Extramissive Involvement} % (fold)
\label{sec:extramission}

It is worth getting clear about what, exactly, it means to describe perception as extramissive. On the basis of our discussion so far, we are in a position to usefully distinguish different grades of extramissive involvement. It is only the fifth and highest grade that incurs a genuine commitment to extramission:
\begin{enumerate}
	\item Perception must at least centrally involve the activity of the perceiver;
	\item This activity is outer directed---in the case of vision, this outer-directed activity is rectilinear;
	\item This outer-directed activity of the perceiver constitutes, at least in part, their perception;
	\item The is outer-directed activity that constitutes, at least in part, the perception of an object involves something spatially extending to the distal object of perception---in the case of vision, along a rectilinear path---so that at least part of the perceiver is substantially located where the the perceived object is or is, at the very least, contiguous with it;
	\item To be perceptible is to be palpable, the perceived object must be in contact with the principle governing sensation.
\end{enumerate}

The first grade is far too weak to incur genuine commitment to extramission. Perception is not even identified with activity but is only claimed to centrally involve activity. Many thinkers accept that perception at least centrally involves activity without accepting, as well, the extramission theory. 

The first two grades, considered jointly, are insufficient for extramission. Taken together they are equivalent to the claim that vision centrally involves rectilinear, outer-directed activity. So understood, they might reasonably be taken to jointly describe looking and seeing. It is not implausible to think that in order to visually perceive an external scene, the perceiver must look at that scene, where looking involves directing one's visual awareness to that scene. Looking, so conceived, is a outer-directed activity of the perceiver that is rectilinear. It determines a line of sight. Plausibly though it may be, the principle---to see, one must look---is a substantive claim that not all may endorse. And yet it falls short of the extramission theory. Notice that the first two grades jointly capture Augustine's claim when he writes ``Sight extends itself outward and through the eyes darts forth far in every possible direction to light up what we see'' (Augustine, \emph{De quantitate animae}, 23 43; \citealt[66]{Colleran:1949ys}). Notice, as well, that in determining lines of sight, the first two grades form a sufficient basis for geometrical optics of the kind developed by Euclid, Hero, and Ptolemy. 

The third grade introduces a further substantive commitment. One may accept the principle that to see, one must look, and yet deny that looking constitutes, even in part, seeing. So the third grade is a further commitment. But even the three grades taken together are insufficient for a commitment to extramission. Contemporary enactivists, for example, such as \citet{Noe:2004fk}, accept something like the first three grades, but enactivism is not a species of extramission.

It can seem that the fourth grade of extramissive involvement is what incurs a genuine commitment to extramission. It is only when the outer-directed activity of the perceiver that constitutes, at least in part, their perception of the object is conceived as something spatially extending to the distal object so that it is in contact with that object do we get a genuine commitment to extramission. But perhaps that is overhasty. 

Francisco Suarez (\emph{Commentaria una cum questionibus in libros Aristotelis de anima} 3.17.1) marks a distinction between accounts where vision occurs at the point where the ray meets the perceived object and accounts where the object's species must somehow be reflected back to the eye first. Suarez cites Galen, in \emph{De Placitis Hippocratis et Platonis} 7, as attributing the latter kind of account to Plato (see, as well, Theophrastus, \emph{De sensibus} 1.5 for a similar reading of Plato). The latter accounts, while involving extramissive elements, seem more aptly deemed interactionist, as vision is the result of the interaction of the perceiver's activity and the activity of the object perceived (\citealt[22--23]{Smith:1996sh}, \citealt{Ierodiakonou:2005ly}, \citealt{Remes:2014en}, and \citealt{Squire:2016aa}). Such accounts not only involve extramissive elements but intromissive elements as well. Moreover, if the species must somehow be reflected back to the eye before vision occurs, then the extramissive activity is merely a precondition for intromission. So the latter accounts are not purely extramissive. Only the former accounts count as purely extramissive. If that is right, then the fourth grade may be accepted without genuine commitment to the extramission theory.

Consider, then, the former, purely extramissive, accounts where vision occurs at the point where the ray meets the perceived object. Perception is understood to be at least modelled on, if not a form of, sensation by contact. Its principle is: to be perceptible is to be palpable. It is the fifth and highest grade of extramissive involvement that generates the requirement that at least a part of the perceiver spatially extend to the distal object of perception. Only in this way could it be palpable to perception and so perceived. Of course, the latter, interactionist accounts also generates that requirement, but they generate that requirement on the basis of different explanatory principles. So it is the fifth and highest grade of extramissive involvement that incurs a genuine commitment to the extramission theory. 

Unclarity about the commitments of the extramission theory is aided and abetted, in certain circumstances, by the application of a certain methodological stricture. Some modern commentators have marked a distinction between the psychological and the physiological claims that Augustine makes. Moreover, there is a subsequent tendency to interpret Augustine's commitment to extramission as a physiological claim about vision. So understood, the extramission theory is simply a false causal model of distal perception and may be dismissed as a piece of antiquated physiology.

Allow me to make two observations about this. First, in discussions of the soul in late antiquity, psychological and physiological issues are intertwined, which is not to say confused. Consideration of the psychological is not so easily separated from the physiological. Second, and more fundamentally, there is more to the extramission theory than a false causal model. The extramission theory, in its fifth and highest grade, essentially involves a psychological claim, that the perceived object must be in contact with the principle governing sensation. And the first three grades of extramissive involvement make important claims about the active, outer-directed phenomenology of vision. Thus the psychologists \citet{Winer:1996as} hypothesize that the tendency for extramission beliefs to persist into adulthood and their resistance to experimental intervention are partly explained by a phenomenological truth enshrined in extramission models.
% :
% \begin{quote}
% 	We assume that core aspects of the phenomenology of vision underlie extramission interpretations. Consider one phenomenologically salient aspect of vision, namely, its orientational or outer-directed quality. When people see, they are generally oriented toward an external visual referent, that is, they direct their eyes and attention to an object in order to see it. In fact, this quality of vision is reflected in language. People talk about ``looking at'' things, and English has expressions such as ``looking out of a window'' and ``looking out of binoculars.'' \ldots\ \citep[140]{Winer:1996as}
% \end{quote}
It would be a mistake to dismiss the extramission theory merely on the grounds of being an antiquated physiology. The extramission theory essentially involves a principle governing sensation and makes important claims about the phenomenology of vision.


% section extramission (end)

\section{Extramission in the Augustinian \emph{Corpus}} % (fold)
\label{sec:extramission_in_the_augustinian_emph_corpus}

In this final section we shall briefly review the strength of evidence for attributing an extramission theory to Augustine on the basis of his other works. At best, it is neither required by scripture nor reason, but is authoritative opinion, and so a defeasible commitment, by Augustine's lights. At times he complains of its subtlety and obscurity, emphasizing its failure to be clearly demonstrated. Many passages involving extramissive elements are, in fact, making a point about something other than perception. And very often this point is independent of the truth of the extramission theory. And sometimes Augustine will pursue his point even when it proves to be in tension with the extramission theory. I conclude that the extramission theory is not a central element in Augustine's thinking about perception. That he was nonetheless persistently drawn to it is explicable. And not just because it was the authoritative opinion of Plato, Ptolemy, and Galen, at least as traditionally understood (we have entertained doubts whether Plato's account is purely extramissive). Rather, perception is the soul's activity using the eye as an instrument. Augustine understands this in a way that commits him to the first three grades of extramissive involvement. 

\subsection{\emph{De musica}} % (fold)
\label{sub:de_musica}

In \emph{De musica}, Augustine compares the way in which the activity of memory comprehends the temporally distant to the way in which the activity of sight, as the extramission theory conceives of it, comprehends the spatially distant:
\begin{quote}
	Then, as the diffusion of rays shining out into the open from tiny pupils of the eye, and belonging therefore to our body, in such a way that, although the things we see are placed at a distance, they are yet quickened by the soul, so, just as we are helped by their effusion in comprehending place-spans, the memory too, because it is somehow the light of time-spans, so far comprehends these time-spans as in its own way ot to can be projected. (\emph{De musica} 6 8 21; \citealt[346]{Taliaferro:1947aa})
\end{quote}
Augustine's point here is not about perception but memory, that it is somehow the light of time-spans that discloses the temporally distant. That point can be made, and made as Augustine does by analogy with the extramission theory of vision, without commitment to the truth of the extramission theory. Indeed, Augustine's doctrine of illumination is more directly relevant to understanding the way in which memory is the light of time-spans that discloses the temporally distant than the extramission theory (on the doctrine of illumination see \citealt{Allers:1952os} and \citealt{Matthews:2014aa}).

% subsection subsection_name (end)

\subsection{\emph{De Genesi ad litteram libri duodecim}} % (fold)
\label{sub:de_genesi_ad_litteram_libri_duodecim}

In \emph{De Genesi ad litteram libri duodecim}, Augustine provides a phenomenology of focal attention that makes use of extramissive elements:
\begin{quote}
	The shaft of rays from our eyes, to be sure, is a shaft of light. It can be pulled in when we focus on what is near our eyes and sent forth when we fix on objects at a distance. But when it is pulled in, it does not altogether stop seeing distant objects, although, of course, it sees them more obscurely than when it focuses its gaze upon them. Nevertheless, the light which is in the eye, according to authoritative opinion, is so slight that without the help of light from outside we should be able to see nothing. Since, moreover, it cannot be distinguished from the outside light, it is difficult as I have said, to find an analogy by which we might demonstrate the diffusion of light to make the day and a contraction to make the night. (\emph{De Genesi ad litteram libri duodecim} 1 16; \citealt[37--38]{Taylor:1982aa})
\end{quote}
The present extramissionist account is derived neither from scripture nor reason, but is accepted as received authoritative opinion. It is potentially revisable in the way that the deliveries of scripture and reason are not. Especially since the authoritative opinion is no mere record of observation. The authoritative opinion includes Plato's explanation, in the \emph{Timaeus} 45b−c, of the necessity of light for seeing, that the light emitted from the eye must be supplemented by external light in order for the perceiver to see. This has the consequence that the emitted light cannot be distinguished from the external light and so is not directly observable. Nevertheless Augustine's acceptance of extramission here, while defeasible and not directly based on observation, seems genuine. 

A tension in the account raises a potential difficulty, however. Extramission theories are motivated by an apparent need to be in contact with distant sense objects if the perceiver is to be aware of them. But Augustine explicitly denies this in this passage ``when it is pulled in, it does not altogether stop seeing distant objects, although, of course, it sees them more obscurely than when it focuses its gaze upon them.'' The visual ray need not be in contact with the distal object in order to perceive it, but it does so less clearly than if it were. The passage provides a phenomenology of focal attention that seems to conflict with the requirement that the principle of sensation be in contact with the objects of sensory awareness.

Perhaps the present account is usefully compared to the extramissionist dioptrics that Nemesius attributes to the ``geometricians'' in \emph{De natura hominis}:
\begin{quote}
	Geometricians draw cones which are formed from the intersection of the rays sent out through the eyes. For they say that the eyes send out rays, the right eye to the left, the left eye to the right, and as a result a cone is formed by their intersection, which is why sight that can encompass many visible things all at once, but sees exactly only those parts where the rays intersect. This is at any rate how, when looking at the floor, we often do not see the coin lying there, though looking hard, until the intersection of the rays falls upon that part where the coin lies and then we gaze upon it as if were were then first paying attention. (\emph{De natura hominis} 7; \citealt[104--105]{Sharples:2008aa})
\end{quote}
We get a similar description of the phenomenology of focal attention, also set within an extramissionist account, but where the dioptric character of vision is made explicit and exploited in an explanation for how we can see many things all at once. For consider a ray sent from a single eye. It would be natural to expect that it sees only that with which it is in contact. But what we see is not restricted in this way. Reflection on binocular vision provides an explanation. The rays from both eyes form a cone. Where the rays intersect is the point of focal attention where things appear exactly in a way that is meant to be consistent with many other things appearing as well if not exactly. However, even objects outside of the cone are in contact with visual rays emitted from at least one eye. So an extramissionist dioptrics would resolve the difficulty raised by the account in \emph{De Genesi ad litteram libri duodecim}. However beyond speaking of the eyes in the plural there is no explicit discussion of dioptrics in that work. And though Augustine's claims about focal attention generated the tension with the extramission theory, Augustine does not acknowledge this tension, let alone consider its resolution by means of an extramissionist dioptrics.

% subsection de_genesi_ad_litteram_libri_duodecim (end)

\subsection{\emph{Sermon 277}} % (fold)
\label{sub:sermon_277}

In 411 CE, on the birthday of the martyr Vincent, Augustine delivered the following as part of a sermon:
\begin{quote}
	In this very body, which we carry around with us, I can find something whose inexpressible swiftness astonishes me; the ray from our eye, with which we touch whatever we behold. What you see, after all, is what you touch with the ray from your eye. (\emph{Sermon} 277 10; \citealt[38]{Hill:1994aa})
\end{quote}
Augustine is using the familiar tactile metaphor associated with extramission theories. But what is it a metaphor for? Arguably, touch is a metaphor for the presentation in sight of the object of vision. That we touch whatever we behold may be too weak, by itself, to establish that, but it is combined with the claim that what you see is what you touch with the visual ray. On this reading, visual presentation is either reduced to or is at the very least modelled on tactile presentation. If that is right, then \citet[46 n.17]{Hill:1994aa} is wrong to speculate that ``presumably on meeting a visible object <the rays> send back the message to the subject...or perhaps they bounce straight back to the eye like radar''. Recall Suarez's contrast (section~\ref{sec:extramission}) between accounts where vision occurs at the point where the ray meets the perceived object and accounts where the object’s species must somehow be reflected back to the eye first. The rays, as Hill conceives of them, are merely part of the causal medium through which information about the perceived object is conveyed. But if the visual rays touch the objects of perception, then they are perceived where they are, and thus there is no need for a signal to return to the subject.

Augustine explains occlusion as the obstruction of visual rays. This could not be an argument for the extramission theory, as occlusion is equally well explained on the intromissionist hypothesis. Rather, the example of a man obscuring a distant column, is setting up the real topic of \emph{Sermon} 277 10, the ``inexpressible swiftness'' of the visual rays, which will lead Augustine to an interpretation of Paul's phrase ``in the twinkling of an eye'' (1 \emph{Corinthians} 15:52) in \emph{Sermon} 277 11, understood as the speed at which the body will be resurrected.

Augustine speaks of the ``inexpressible swiftness'' of the visual rays, but, strictly speaking, their action is instantaneous. If two objects, one near---a man---and one far---a column---are visible to the perceiver, in the circumstances of perception (and hence the man no longer occludes the column), then it is not the case that the visual rays reach the near object sooner than the far: 
\begin{quotation}
	You don't get to him sooner and to it later; and here he is, nearby, and it's a long way off. If you wanted to walk, you would get to the man sooner than to the column; because you wanted to see, you have got to the column as soon as the man. 
	
	And yet, as soon as you open your eyes, lo and behold, you yourself are here, your ray is there. As soo as you wanted to see it, you reached it by seeing it. \ldots\ Just opening your eyes constitutes reaching it. (\emph{Sermon} 277 10; \citealt[39]{Hill:1994aa})
\end{quotation}

Notice that the ``inexpressible swiftness'' of the visual rays is contrasted with the speed of corporeal processes such as walking. Perhaps, like Philoponus, Augustine maintains that instantaneous action at a distance is only possible for incorporeal activity (\emph{In de anima} 325 1-341 9). After all, instantaneous action at a distance would require a body to travel at infinite speed, but bodies, no matter how swift, only travel at finite speeds. Galen's version of the extramission theory resolves this difficulty by denying that what travels with ``inexpressible swiftness'' is a body (thus also resolving some of Aristotle's objections in \emph{De sensu} 2). But neither is it an incorporeal activity as Philoponus and perhaps Augustine maintain. Rather it is an effect of the emitted \emph{pneuma}, a quality immediately instantiated by the external light. Like Aristotle, Galen maintains that the external light is already a continuous unity. The ``inexpressible swiftness'' would consist in the quality being instantiated, all at once, by that continuous unity (\emph{De Placitis Hippocratis et Platonis} 7.4--5). There is no evidence that Augustine entertained, let alone endorsed, the Galenic alternative.

% The presence of that quality sensitizes the external light making it an instrument of the \emph{pneuma} flowing from the optic nerve, just as the \emph{pneuma} in the nerves is the instrument of the \emph{pneuma psychikon} in the brain, the seat of sensation and movement in the living body .

% The contrast that Augustine draws between the speed of the eyelid in opening one's eyes and the speed of the visual rays thereby unleashed is also relevant:
% \begin{quote}
% 	The twinling of an eye does not consist in closing and opening the eyelids, because this is done more slowly than seeing. You bat an eyelid more slowly than you direct a ray. Your ray gets to the sky more quickly than the batted eyelids reach the eyebrow. (\emph{Sermon} 277 11; \citealt[39]{Hill:1994aa})
% \end{quote}
% Notice that Augustine identifies the speed of the visual rays with the speed of seeing, providing further evidence for the hypothesis that visual presentation is being understood as a kind of touch by visual rays.


% subsection subsection_name (end)

\subsection{\emph{De Trinitate}} % (fold)
\label{sub:_emph_de_trinitate}

In \emph{De Trinitate}, Augustine writes:
\begin{quote}
	For the mind does not know other minds and not know itself, as the eye of the body sees other eyes and does not see itself; for we see bodies through the eyes of the body, because, unless we are looking into a mirror, we cannot refract and reflect the rays themselves which shine for through the eyes, and touch whatever we discern---a subject, indeed, which is treated of most subtly and obscurely, until it be clearly demonstrated whether the fact be so, or whether it be not. But whatever is the nature of the power by which we discern through the eyes, certainly, whether it be rays or anything else, we cannot discern with the eyes that power itself; but we inquire into it with the mind, and if possible, understand this with the mind. (\emph{De Trinitate} 9 3; \citealt[226]{Haddan:1873aa})
\end{quote}
Again, the present extramissionist account is derived neither from scripture nor reason. Nor is it, in this instance, even accepted as received authoritative opinion. Notice that Augustine, after having introduced the extramissionist imagery of rays, immediately brackets that commitment, claiming that it is treated subtly and obscurely and claims that the explanation of perceptual discernment by rays has not yet been clearly demonstrated. The central point of this passage is independent of the truth of the extramission theory. Moreover, it echoes a Neoplatonic theme. We get a contrast between the activity of the mind and the activity of the animated eye. Whereas the mind is spiritual, the animated eye is a compound of the corporeal and the spiritual. The animated eye is part of the living being, the compound of the organ and the sensitive soul that animates that organ and uses it as an instrument for the soul's activities. The mind may apprehend itself in thought in the way that the power of sight acting through the eye could not apprehend itself in vision.  In Neoplatonic vocabulary, the mind's activity, being purely spiritual, is capable of reverting upon itself the way that the visual activity of the animated eye, being a compound of the corporeal and the spiritual, could not (compare, looking back, to Proclus' demonstration of proposition 15 of \emph{Elementatio Theologica}: “All that is capable of reverting upon itself is incorporeal” and, looking forward, to Aquinas' \emph{Super Librum de Causis expositio} propositions 7 and 15). Augustine's point about the failure of sight's activity to revert upon itself does not depend upon the truth of extramission. Augustine's point is, in that sense, independent of the truth of the extramission theory.

% subsection _emph_de_trinitate (end)

% section extramission_in_the_augustinian_emph_corpus (end)

\section{Conclusion} % (fold)
\label{sec:conclusion}

The extramission theory is not a central element in Augustine’s thinking about perception. That he was nonetheless persistently drawn to it is explicable. And not just because it was the authoritative opinion of Plato, Ptolemy, and Galen, at least as traditionally understood. Rather, perception is the soul’s activity using the eye as an instrument. Perception is not something done to the perceiver, it is the soul, through the use of the body, that perceives. Sensory awareness is an activity of the soul within the soul itself. According to Augustine, vision involves outer-directed, rectilinear activity that constitutes the perception of the object. Augustine is committed to the first three grades of extramissive involvement. But a genuine commitment to extramission is only incurred with the acceptance of the fifth and highest grade. And in \emph{De quantitate animae} at least, Augustine is committed to its rejection.

% section conclusion (end)





% no cite
\nocite{Tourscher:1933rw}
\nocite{Dodds:1963ul}
\nocite{Migne:1845aa}
\nocite{Migne:1856aa}
\nocite{Lacy:1980mk}
\nocite{Castellote:1978qe}
\nocite{Sorabji:1997ly}
\nocite{Guagliardo:1996aa}
\nocite{Wright:1981zr}
\nocite{Inwood:2001ve}

%Bibliography
\bibliographystyle{plainnat}
\bibliography{Philosophy}

\end{document}
