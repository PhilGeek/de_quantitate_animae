%!TEX TS-program = xelatex 
%!TEX TS-options = -synctex=1 -output-driver="xdvipdfmx -q -E"
%!TEX encoding = UTF-8 Unicode
%
%  de_quantitate_animae
%
%  Created by Mark Eli Kalderon on 2016-01-09.
%  Copyright (c) 2016. All rights reserved.
%

\documentclass[12pt]{article} 

% Definitions
% \newcommand{\change}{\textcolor{blue}{\textbf{CHANGE SLIDE}}}
\newcommand\mykeywords{Augustine, vision, perception, extramission}
\newcommand\myauthor{Mark Eli Kalderon}

% Packages
\usepackage{geometry} \geometry{a4paper} 
\usepackage{url}
% \usepackage{txfonts}
\usepackage{color}
\usepackage{enumerate}
\definecolor{gray}{rgb}{0.459,0.438,0.471}
\usepackage{setspace}
% \doublespace % Uncomment for doublespacing if necessary
% \usepackage{epigraph} % optional

% XeTeX
\usepackage[cm-default]{fontspec}
\usepackage{xltxtra,xunicode}
\defaultfontfeatures{Scale=MatchLowercase,Mapping=tex-text}
\setmainfont{Hoefler Text}
\newfontfamily{\sbl}{SBL Greek}

% Bibliography
\usepackage[round]{natbib}

% Title Information
\title{Perception in \emph{De quantitate animae}}
\author{\myauthor} 
\date{} % Leave blank for no date, comment out for most recent date

% PDF Stuff
\usepackage[plainpages=false, pdfpagelabels, bookmarksnumbered, backref, pdftitle={Form Without Matter}, pagebackref, pdfauthor={\myauthor}, pdfkeywords={\mykeywords}, xetex, colorlinks=true, citecolor=gray, linkcolor=gray, urlcolor=gray, unicode=true]{hyperref} 

%%% BEGIN DOCUMENT
\begin{document}

% Title Page
\maketitle
\begin{abstract}
	\noindent Augustine is commonly interpreted as endorsing an extramission theory of perception in \emph{De quantitate animae}. A close examination of the text shows, instead, that he is committed to its rejection. I end with some remarks about what it takes for an account of perception to be an extramission theory.
\end{abstract}

% Layout Settings
\setlength{\parindent}{1em}

% Main Content

\section{Augustine and Extramission} % (fold)
\label{sec:augustine_and_extramission}

Augustine is commonly interpreted as endorsing an extramission theory of perception. Extramissive elements can be found in a number of his works (\emph{De musica} 6 8 21, \emph{De Genesi ad litteram libri duodecim} 1 16, \emph{Sermon} 277 10, \emph{De Trinitate} 9 3). However, at least in his early work, \emph{De quantitate animae}, far from endorsing an extramission theory of perception, Augustine explicitly argues for its rejection. And yet on a standard interpretation, Augustine is understood to endorse the extramission theory in \emph{De quantitate animae} (see, for example, \citealt[82--3]{ODaly:1987fq}). It is perhaps worth considering Augustine's anti-extramission argument in detail. The present essay ends with two sets of reflections. First, we shall consider what it means to describe an account of perception as extramissive. We shall do so by distinguishing different grades of extramissive involvement. Part of the diagnosis for the misattribution of an extramission theory has to do with unclarity about the grades of extramissive involvement. Second, we shall briefly review the strength of the evidence for attributing an extramission theory to Augustine on the basis of his other works. We shall see that it is neither required by scripture nor reason, but represents authoritative opinon, such as Galen's, and so is, by Augustine's lights, a defeasible commitment.

% section augustine_and_extramission (end)

\section{The Textual Evidence} % (fold)
\label{sec:the_textual_evidence}

The textual evidence for attributing an extramission theory to Augustine occurs in chapter 23 of \emph{De quantitate animae}. The primary evidence consists in two passages from \emph{De quantitate animae} 23.43, but there is also a back reference to the second passage at the beginning of \emph{De quantitate animae} 23.44.

The first passage is as follows:
\begin{quote}
	Sight extends itself outward and through the eyes dart forth in every possible direction to light up what we see. (\emph{De quantitate animae} 23.43, \citealt[66]{Colleran:1949ys}) 
\end{quote}
The second passage involves the stick analogy that Alexander of Aphrodisias attributes ot the Stoics in \emph{De anima} 130 14:
\begin{quote}
	I say that be means of sight, reaching out to that place where you are, I see you where you are. But that I am not there, I admit. Stil let us suppose that I were to touch you with a stick: I certainly would be the one doing the touching and I would sense it; yet I would not be there where I touched you. (\emph{De quantitate animae} 23.43, \citealt[66]{Colleran:1949ys}) 
\end{quote}

One issue concerning either passage is their dialectical context. Both occur in chapter 23 as part of an extended discussion of perception that only culminates in chapter 30. In chapter 23, Augustine presents Evodius with a puzzle about seeing at a distance the full force of which he is only able to appreciate in chapter 30 once certain conceptual obstacles are removed (). Claims made at this stage of the dialogue are unlikely to be definitive. We shall have more to say about this as we proceed.

Let's begin with the first passage. It is not an unambiguous statement of the extramission theory. First, that sight extends itself outwards through the eyes seems like a reasonable description of looking and seeing, at least as a Platonist conceives of it. But the outward activity of looking and seeing is not the exclusive provenance of the extramission theory. The Platonic element of Augustine's description consists in two things, the first more explicit then the next.  First, sight is a power of the soul that is exercised through the use of the eyes. It is an instrument of the soul (\emph{De quantitate animae} 33.41, \emph{De Genesi ad litteram libri duodecim} 12.24.51). That the body is an instrument of the soul is a distinctively Platonic thought (compare Plotinus, \emph{Ennead} 4.7.8). Second, Augustine is here emphasizing how seeing is an activity of sight, a power of the soul. The superiority of the incorporeal soul is manifest in its ability to act upon the sensible and corporeal without their being able to act, in turn, upon the soul (\emph{De musica} 6.5.8--10; see \citealt{Silva:2014bh} for discussion). So it is the soul that acts in seeing and so places itself in the distal body seen (\citealt[205, n.55]{Colleran:1949ys}). The emphasis here is on the activity of the soul as opposed to its being substantially located where the perceived object is,

The outward activity of looking and seeing is not the exclusive provenance of the extramission theory. One might object that this ignores the illumationist imagery at the end of this passage---sight darts forth from the eyes and illuminates what it sees. While it is true that the visual ray of the extramission theory is often likened to light, the illuminationist imagery, considered by itself, is insufficient grounds for the attribution of the extramission theory to Augustine. Other thinkers, who have explicitly rejected the extramission theory, have coherently embraced this imagery. The illuminationist imagery is undoubtedly of Neoplatonic origin, but neither Plotinus nor Porphyry are extramission theorists. Nor are Iamblichus, Proclus, Priscian, and Pseudo-Simplicius. In addition, later scholastic thinkers influenced by Augustine employ the Neoplatonic illuminationist imagery while disavowing the extramission theory. Thus Peter John Olivi compares a perceiver's gaze to light illuminating its object (\emph{Questiones in secundum librum Sententiarum} q. 72 35–36) even as he denies that seeing involves any real emission (\emph{Questiones in secundum librum Sententiarum} q. 58 ad 14.8).

% section the_textual_evidence (end)

\section{Dialectical Context} % (fold)
\label{sec:dialectical_context}

The force of the textual evidence for attributing an extramission theory to Augustine, in \emph{De quantitate animae}, crucially depends upon the dialectical context. Allow me to briefly review the place of Augustine's account of perception in that dialogue.

\emph{De quantitate animae}  is mainly charged with the task of arguing for the incorporeal nature of the living soul. In the dialogue, Evodius, like Augustine’s former self (\emph{Confessions} 7.1ff), has a hard time conceiving of something that is both real and incorporeal (compare the position of the Giants in Plato's \emph{Sophist}). Throughout \emph{De quantitate animae}, Augustine will give accounts of the soul’s activities and powers that are meant to persuade us that these are not activities and powers of the body. The question of the soul’s incorporeal nature is linked with the question of its magnitude. Bodies are extended in three dimensions. If souls are inextended, if they lack extensive magnitude, then they are incorporeal. But, importantly, being incorporeal is consistent with the soul’s possession of superior virtual magnitude. That is to say that psychic powers, the powers and virtues of the soul, are superior to any corporeal power. 

The question concerning the magnitude of the soul is subject to further specification since two senses of magnitude may be distinguished (DQA 3.4):
\begin{enumerate}
	\item extensive magnitudes: magnitudes of extension, “How tall is Hercules?”
	\item virtual magnitudes: magnitudes of power, “How great is Hercules valour or prowess?”
\end{enumerate}
Evodius seeks an answer to the question in both senses. In the sense of extensive magnitude, Augustine denies that the soul has quantity at all. The soul is inextended, and, hence, incorporeal (since corporeal bodies are necessarily extended in three dimensions). Augustine will maintain that the soul, while lacking extensive magnitude, nevertheless possesses virtual magnitude. Whereas the question how great is the soul in the sense of extensive magnitudes is answered in the negative in \emph{De quantitate animae} 3.4, a full answer to the question how great is the soul in the sense of virtual magnitude, in its powers and virtues, only emerges in the hierarchically organised enumeration of the soul’s powers that ends the dialogue (\emph{De quantitate animae} 33–36). This hierarchically organised enumeration of the soul’s powers is also, at the same time, a soteriology, at least in part, in that it describes the soul’s ascent to God (and it is in this sense that it is a theoretical articulation of the vision in Ostia that Augustine shared with Monica as reported in the \emph{Confessions}).

Evodius will resist Augustine’s denial that the soul possesses extensive magnitude. Extensive magnitudes cited by Augustine are length, width, and strength. Strength here translates \emph{robustam} which means the resistance offered by solid space-filling things. Specifically, then, Evodius doubts whether something without length, width, and strength so much as could exist. Here Evodius is echoing the position of the Giants. In the \emph{Sophist}, Plato re-envisions the Gigantomachy, the struggle for political supremacy of the cosmos between the Giants and the Olympian Gods, as a metaphysical dispute between corporealists, the Giants, and the Friends of the Forms, the Gods. Compare Evodius position to the Giants:
\begin{quote}
	One party is trying to drag everything down to earth out of heaven and the unseen, literally grasping rocks and trees in their hands, for they lay hold upon every stock and stone and strenuously affirm that real existence belongs only to that which can be handled and offers resistance to the touch. (Plato, \emph{Sophist} 246a; Cornford in \citealt[990]{Hamilton:1989fk})
\end{quote}
	

In response, Augustine will offer a negative argument and a positive argument. According the negative argument, just because the soul lacks extension does not mean that it is not real (\emph{De quantitate animae} 3.4–4.5). And according to the positive argument, the soul must be incorporeal since it possesses powers that bodies lack (\emph{De quantitate animae} 4.6–15.25).

In the \emph{Sophist}, the Eleatic Visitor convinces the Giants to modify their corporealism in order to allow for justice, since the denial of this virtue would be impious. Justice lacks length, width, and strength. It cannot be grasped and offers no resistance to touch (an hence lacks strength, \emph{robustam}). And it is by means of the Eleatic Visitor’s argument that Augustine convinces Evodius that lacking extensive magnitude does not entail nonexistence. Specifically, a tree, a sensible and corporeal object with extensive magnitude exists. But so does justice despite lacking extensive magnitude. Moreover, and importantly, justice is greater than the tree. The adaption of the Eleatic Visitor’s argument is meant to establish not only that virtues like justice may exist despite being unexpended but that they may also be more excellent than any extended thing. Justice may lack extensive magnitude and yet possess greater virtual magnitude than a sensible body. 

Augustine may have established that the soul’s nonexistence does not follow from its lack of extension, but he has yet to establish what the soul positively is. Moreover, just because justice is real despite being inextended, it does not follow that the soul itself is inextended. Evodius is persistently attracted to the idea that the soul extends throughout the body that it animates and hence must itself be extended. Augustine will argue, in contrast, that the soul is inextended because it possesses powers that corporeal extended things lack. In effect, Augustine is arguing that the soul possesses greater virtual magnitude than any extended corporeal thing. It is the greatness of the soul that establishes its inextended, and hence, incorporeal nature.

Augustine’s general response is to emphasise how an account of the activities and powers of the soul establish that they are not activities and powers of a body. Indeed, so construed, Augustine offers not one positive argument but two:
\begin{enumerate}
	\item In visually imagining a remembered body, my mental image is not constrained in the way that corporeal images are. Corporeal likenesses are only as large as the body in which the image occurs, incorporeal likenesses are not so constrained (\emph{De quantitate animae} 5.7–9)
	\item In conceiving of geometrical figures abstracted from three-dimensional bodies—such as planes, lines, and points that figure as parts of those bodies—the soul must be incorporeal since only like can conceive of like (\emph{De quantitate animae} 6.10–15.26)
\end{enumerate}

At this point the central case of the dialogue has been made. More specifically, Augustine has completed his negative and positive arguments. That is to say, Augustine has argued that just because something is incorporeal does not mean that it is less real or less valuable than something corporeal. Moreover, Augustine has argued that the soul must be incorporeal because it possesses powers that corporeal extended things lack. And the incorporeal powers of the soul are either more valuable than their corporeal counterparts or at least more valuable than the corporeal objects of their activity. Evodius accepts that Augustine has established this. What are raised are less objections per se to the conclusions of Augustine's negative and postive arguments than certain residual problems or challenges. There are two:
\begin{enumerate}
	\item If the soul is inextended, how is it that the soul grows over time as the body grows? (\emph{De quantitate animae} 15.26–22)
	\item If the soul is inextended, then it is not extended throughout the body. But if the soul is not extended throughout the body, then how can it sense a stimulus on any part of the body? (\emph{De quantitate animae} 23–30)
\end{enumerate}
	
A tacit assumption is at work in Evodius' objection. To bring this out, notice that Evodius objection turns on a particular form of tactile perception, sensation by contact. To see what is distinctive about this consider two contrasts. First sensation by contact contrasts with haptic perception. Haptic perception involves the active exploration of the tangible object and draws upon various forms of bodily awareness such as kinaesthesia and proprioception. Haptic perception necessarily involves temporal duration, the time in which the exploratory activity takes place. Sensation by contact, by contrast, is immediate. Sensation by contact also contrasts with distal touch. Distal touch involves sensing the tangible features of an object that is not in direct contact with the perceiver's body. Thus one can feel the rigidity and texture of a rock by poking it with a stick, or feel the wooden frame of through the padding of a Victorian hobby horse. In neither case is the perceiver's body in direct contact with the object of tactile perception. What one feels is the rock, but the stick is in contact with the rock, not the perceiver's body. And what one feels is the wooden frame even though the perceiver's body is not in contact with it but with the padding. Attending only to sensation by contact, and perhaps by regarding it as an exemplary form of perception, can suggest that the principle governing sensation takes a certain a form. Its slogan might be: the be perceptible is to be palpable. The ideas is the perceived object must be in contact with the principle governing sensation. The sensitive soul is the principle governing sensation. When Evodius is touched upon his body he feels it. That means that the principle of his sensation, Evoidus' sensitive soul, must be in contact with what touches his body, if it is to be perceptible. But it is perceptible. And so the sensitive soul, at least, must be extended throughout the body.

Augustine is sensitive to this tacit assumption even though he does not make it explicit. This sensitivity is manifest in a curious shift in example. Whereas Evodius' objection turns on an observation about touch, understood as sensation by contact, Augustine develops his definition of perception with reference to vision, a form of distal perception. The objects of vision must be at a distance from the perceiver's eyes. Augustine's intent is to undermine the tacit assumption driving Evodius to think that the soul must be extended throughout the body if sensation by contact is to be so much as possible.

% section dialectical_context (end)

\section{The definition} % (fold)
\label{sec:the_definition}

When asked by Augustine what sense perception is, Evodius, good Socratic stooge that he is, merely responds with a list, the Peripatetic five senses---vision, audition, olfaction, taste, and touch. Augustine explains that they are presently seeking a single definition that would encompass all five senses (compare Plato, \emph{Theaetetus} 145e--147c) and proposes a candidate definition for Evodius to defend or reject. This definition will subsequently be refined, though it will retain its basic form. It is worth, however, discussing the initial formulation of the definition: 
\begin{quote}
	sensum puto esse non latere animam quod patitur corpus (Augustine, \emph{De quantitate animae}, 23 41)
\end{quote}
Augustine's provisional definition comes in two parts: \emph{quod patitur corpus} describes the object of perception whereas \emph{non latere animam} describes the soul's relation to that object. 

Consider the soul's relation to the object of perception first. And bracket, for the moment, the significance of Augustine's indirect description of that relation. In describing the object of perception as being not hidden from the soul, Augustine is defining perception, in the first instance, as a mode of awareness \citep[275]{Brittain:2002hl}. 

% What is the significance of Augustine's indirect description of the soul's relation to the object of perception? Why describe the object of sensory awareness as something that is not hidden from the soul? Some have found this worrying. \citet[112]{Bourke:1947jk} flatly pronounces Augustine's definition ``unsatisfactory'' on just this basis: ``The persistent use of the negative formula (\emph{non latere}) indicates Augustine's inability to say positively just what sensation is.'' And Bourke's judgment is reaffirmed by \citet[104, n. 1]{McMahon:1947dn}. However, I wonder to what extent Bourke's judgment is fair. If we accept, as seems evident, that Augustine had in mind a mode of awareness, then perhaps to describe the object of perception as not hidden from the soul is, after all, to provide a positive characterization of the sensory awareness afforded by perceptual experience.
%
% Etienne Gilson thought so:
% \begin{quote}
% 	His purpose is to make sensation an activity of the soul within the soul itself. This is really the reason why he defines it in such a roundabout way. The phrase ``\emph{non latet}'' indicates precisely that the soul is a spiritual force, ever watchful and attentive. In order to sense, it does not have to receive anything from the organ that it vivifies; it is enough if the changes undergone by the organs do not escape its notice, and come within the range of its attention. \citep[63]{Gilson:1961ec}
% \end{quote}
% The positive characterization that Gilson sees in Augustine's roundabout expression is metaphysically significant. We shall return to its significance in due course. But for now, notice how Gilson understands \emph{non latere animam} as the soul's attentive vigilance.
%
% I believe that Gilson was right to see in Augustine's roundabout expression a positive characterization of sensory awareness. Consistent with Gilson's suggestion, though logically independent of it, one might also understand Augustine as claiming that sensory awareness is a mode of disclosure. To describe the object of perception as not hidden from the soul is to understand sensory awareness not only as the soul's attentive vigilance but also as a mode of disclosure---an activity whereby what was previously hidden from the range of its attention is now revealed to the soul.

Consider now Augustine's description of what is not hidden from the soul, the object of sensory awareness. \citet[274--278]{Brittain:2002hl} observes that there is a crucial ambiguity in \emph{quod patitur corpus}. The verb \emph{patior} means to suffer or undergo, to be affected. So a natural understanding of this phrase might be what the body undergoes. So understood, the object of sensory awareness is a bodily affection, the way in which the body is affected. However, \emph{quod patitur} might also be read as Latin rendering of the Greek {\sbl ὅ τι πάσχει}. So understood, the object of sensory awareness is less a bodily affection than what affects the body. On the former understanding, the object of sensory awareness is internal---it is the perceiver's body being affected in a certain way. On the latter understanding, the object of sensory awareness is external, at least if we rule out cases of auto-affection---it is what affects the body from without. 

\citet[]{Brittain:2002hl} observes that understanding \emph{quod patitur corpus} as what affects the body better coheres with Augustine's own examples of the objects of perception in \emph{De quantitate animae}. The objects of vision, for example, are external bodies located at a distance from the perceiver. Thus Augustine sees Evodius, and Evodius sees Augustine, and neither is at the place where the other is. The objects of perception are external to the perceiver's body. In modern parlane, perception is exteroceptive. And this would remain true even if the sensory disclosure of external bodies involves the perceiver's sensory organs being affected, or even the formation of incorporeal images in the soul occasioned by such affections. 

Understanding \emph{quod patitur corpus} as bodily affection, \emph{passio corporis}, yields a definition of what is, at best, bodily sensation. It at best characterizes a form of interoceptive awareness. It is only on the understanding of \emph{quod patitur corpus} as what affects the body that it plausibly yields a definition of perception. Only so understood does the definition characterize a form of exteroceptive awareness. After all, what affects the body, the objects of perception, are external that body, and the objects of visual perception are located at a distance from the perceiver's body.

\citet[]{ODaly:1987fq} suggests that the ambiguity is intentional and that both sensation and perception involve sensory awareness:
\begin{quote}
	Augustine may wish to distinguish here between sensation (\emph{passio corporis}) and perception (\emph{non latere animam}), but it is probably truer to say that he is rather thinking of the external source of most sensory stimuli when he speaks of \emph{passio corporis}, and including under awareness (\emph{non latere}) all forms of sensation and perception.
\end{quote}

If perception fundamentally involves the soul's sensory awareness understood as a mode of disclosure, and if the objects not hidden from the soul are external bodies, then the emphasis of the expression \emph{quod patitur corpus} is on the fact that the soul's sensory awareness of external bodies is mediated by the body's affection. In this way perception contrasts with the understanding. The soul's awareness of the intelligible is not mediated by the body's affection the way that the soul's sensory awareness is.

% section the_definition (end)

\section{Augustine's Counterargument} % (fold)
\label{sec:augustine_s_counterargument}

Augustine counterargument begins with a restatement of the refined definition whose validity has been accepted by Evodius and Augustine:
\begin{quote}
	Perception is what affects the body by itself not being hidden from the soul
\end{quote}
From the accepted definition it immediately follows that:
\begin{quote}
	If one perceives, one's body is acted upon
\end{quote}
Moreover, it was earlier agreed that:
\begin{quote}
	Seeing is a form of perception
\end{quote}
But seeing is in some ways a distinctive form of perception. It is a mode of distal perception. Importantly, it was also agreed that:
\begin{quote}
	A perceiver sees an object where the perceiver is not
\end{quote}
If sight was confined to where the perceiver is at, at best the eye alone could be seen. (And not even the eye, if Aristotle in \emph{De anima} is to be believed.) But the objects of sight are located at a distance from the perceiver. So the perceiver sees an object where they are not. 

From these premises a startling conclusion follows. Seeing is a form of perception. If one perceives, one is acted upon. So in seeing one is acted upon. And since what acts upon one when one sees is located where one is not, in seeing one is acted upon where one is not. The conclusion involves attributing to the animate eye a passive power to be affected where it is not. This is a passive power not shared with soulless natural bodies. Inanimate natural bodies can only be acted upon by what is in contact with them. The living eye, part of a whole and healthy animal, is animated by the sensitive soul, and it is a manifestation of the superiority of the soul that it endows the eye that it animates with the passive power to be affected where it is not. The sensitive soul may lack extensive magnitude, but it has in this way great virtual magnitude. This nicely fits the pattern we observed in Augustine's postive argument for the intextended nature of the soul. This involved attributing superior powers to the soul not possessed by extended corporeal things. Thus while corporeal images are limited by the size of the body upon which they are inscribed, the soul's power to recall an image of an object previously seen is not subject to this limitation. Similarly, the soul possesses the power to conceive of incorporeal geometrical abstractions, and since only like may conceive of like, the soul itself must itself be incorporeal. It thus possesses a power that no corporeal thing may have. The passive power to be affected where one is not conferred by the sensitive soul is similarly a power that no soulless body may possess. Further testimony to the greatness of soul, conceived, not as greatness of extent but greatness of power or virtue.

How does this attribution of a passive power possessed by no corporeal thing bare on Evodius' objection? Recall the tacit assumption behind Evodius' use of sensation by contact, that the object of sensation must be in contact with its principle, that by which one senses. It is just this assumption that drives Evodius conviction that the soul must be extended throughout the body. For only if it were could it be in contact with what touches the body. But if the sensitive soul confers the passive power to be affected where one is not, then there is no need for the sensitive soul to extend throughout the body. The soul need not be where it is affected for this affection to be not hidden from the range of its attention. Though the writing has been on the wall since chapter 23, Evodius is stunned:
\begin{quote}
	That conclusion upsets me very much, so much, in fact, that I am completely stunned. I do not know what to answer and I do not know where I am. What shall I say? Shall I say that a bodily experience of which the soul is aware directly is not sensation? What is it, then, if it is not that? Shall I say that the eyes experience [i.e., that the eyes are affected by] nothing when we see? That is most absurd. Shall I say that the eyes experience where they are? But they do not see themselves and nothing is where they are, except themselves. Shall I say that the soul is not more powerful than the eyes, when the soul is the very power of the eyes? Nothing is more unreasonable. Or must this be said, that it is a sign of greater power to experience there where something is than to experience it where it is not? But, if that were true, sight would not be rated higher than the other senses.
\end{quote}
The reason that corporeal things lack the passive power to be affected where they are not is because, being corporeal and extended, they are confined to the place where they are. The soul, being inextended and not so confined, may confer this passive power on a body that it animates.

% section augustine_s_counterargument (end)

\section{Extramission} % (fold)
\label{sec:extramission}

It is worth getting clear about what, exactly, it means to describe perception as extramissive. It is useful in this regard to distinguish different grades of extramissive involvement. It is only the fourth and highest grade that incurs a genuine commitment to extramission:
\begin{enumerate}[(1)]
	\item Perception, so conceived, must at least centrally involve the activity of the perceiver;
	\item This activity is outer directed---in the case of vision, this outer-directed activity is rectilinear;
	\item This outer-directed activity of the perceiver constitutes, at least in part, their perception;
	\item The is outer-directed activity that constitutes, at least in part, the perception of an object involves something spatially extending to the distal object of perception---in the case of vision, along a rectilinear path---so that at least part of the perceiver is substantially located where the the perceived object is or is, at the very least, contiguous with it.
\end{enumerate}

The first two grades, considered in and of themselves, are insufficient for extramission. Indeed, they might reasonably be taken to jointly describe looking. It is not implausible to think that in order to visually perceive an external scene, the perceiver must look at that scene, where looking involves directing one's visual awareness to that scene. Looking, so conceived, is a outer-directed activity of the perceiver that is rectilinear. It determines a line of sight. Plausibly though it may be, the principle---to see, one just look---is a substantive claim that not all may endorse. And yet it falls short of the extramission theory. Notice that the first two grades jointly capture Augustine's claim when he writes ``Sight extends itself outward and through the eyes darts forth far in every possible direction to light up what we see'' (Augustine, \emph{De quantitate animae}, 23 43; \citealt[66]{Colleran:1949ys}). Notice, as well, that in determining lines of sight, the first two grades form a sufficient basis for geometrical optics of the kind developed by Euclid, Hero, and Ptolemy. 

The third grade introduces a further substantive commitment. One may accept the principle that to see, one must look, and yet deny that looking constitutes, even in part, seeing. So the third grade is a further commitment. But even the three grades taken together are insufficient for a commitment to extramission. Contemporary enactivists, for example, such as \citet{Noe:2004fk}, accept something like the first three grades, but enactivism is not a species of extramission.

It is only the fourth and highest grade of extramissive involvement that incurs a genuine commitment to extramission. It is only when the outer-directed activity of the perceiver that constitutes, at least in part, their perception of the object is conceived as something spatially extending to the distal object so that it is in contact with that object do we get a genuine commitment to extramission. Moreover, it is only with the fourth grade that the principle governing extramissive perception becomes evident. Perceptual presentation is understood to be at least modelled on if not a form of sensation by contact. Its principle is: to be perceptible is to be palpable \citep[see][for discussion]{Kalderon:2015fr}.

% section extramission (end)

\section{Extramission in the Augustinian \emph{corpus}} % (fold)
\label{sec:extramission_in_the_augustinian_emph_corpus}

\subsection{\emph{De musica}} % (fold)
\label{sub:subsection_name}

\begin{quote}
	Then, as the diffusion of rays shining out into the open from tiny pupils of the eye, and belonging therefore to our body, in such a way that, although the things we see are placed at a distance, they are yet quickened by the soul, so, just as we are helped by their effusion in comprehending place-spans, the memory too, because it is somehow the light of time-spans, so far comprehends these time-spans as in its own way ot to can be projected. (\emph{De musica} 6 8 21; Robert Catesby Taliaferro, in The Fathers of the Church, volume 4, St  Augustine, The Immortality of the Soul, The Magnitude of the Soul, On Music, The Advantage of Believing, On Faith in Things Unseen, The Catholic University of America Press, 1947, 346)
\end{quote}

% subsection subsection_name (end)

\subsection{\emph{De Genesi ad litteram libri duodecim}} % (fold)
\label{sub:de_genesi_ad_litteram_libri_duodecim}

\begin{quote}
	The shaft of rays from our eyes, to be sure, is a shaft of light. It can be pulled in when we focus on what is near our eyes and sent forth when we fix on objects at a distance. But when it is pulled in, it does not altogether stop seeing distant objects, although, of course, it sees them more obscurely than when it focuses its gaze upon them. Nevertheless, the light which is in the eye, according to authoritative opinion, is so slight that without the help of light from outside we should be able to see nothing. Since, Moreover, it cannot be distinguished from the outside light, it is difficult as I have said, to find an analogy by which we might demonstrate the diffusion of light to make the day and a contraction to make the night. (\emph{De Genesi ad litteram libri duodecim} 1 16; John Hammond Taylor, S.J., *St Augustine, The Literal Meaning of Genesis*, New York NY and Mahway NJ: Paulist Press, 1982, 37--38)
\end{quote}
The present extramissionist account is derived neither from scripture nor reason, but is accepted as received authoritative opinion. It is potentially revisable in the way that the deliveries of scripture and reason are not. Nevertheless Augustine's acceptance of extramission here, while defeasible, seems genuine.

A tension in the account raises a potential difficulty, however. Extramission theories are motivated by an apparent need to be in contact with distant sense objects if the perceiver is to be aware of them. But Augustine explicitly denies this in this passage ``when it is pulled in, it does not altogether stop seeing distant objects, although, of course, it sees them more obscurely than when it focuses its gaze upon them.'' The visual ray need not be in contact with the distal object in order to perceive it, but it does so less clearly than if it were. The passage provides a phenomenology of focal attention that seems to conflict with the requirement that the perceiver be in contact with the objects of sensory awareness.

Perhaps the present account is usefully compared to the extramissionist dioptrics that Nemesius attributes to the ``geometricians'' in \emph{De hominis} 7:
\begin{quote}
	> Geometricians draw cones which are formed from the intersection of the rays sent out through the eyes. For they say that the eyes send out rays, the right eye to the left, the left eye to the right, and as a result a cone is formed by their intersection, which is why sight that can encompass many visible things all at once, but sees exactly only those parts where the rays intersect. This is at any rate how, when looking at the floor, we often do not see the coin lying there, though looking hard, until the intersection of the rays falls upon that part where the coin lies and then we gaze upon it as if were were then first paying attention. (\emph{De natura hominis} 7; R.W. Sharples and P.J. van der Eijk, *Nemesius On the Nature of Man*, Liverpool University Press, 2008, 104--105)
\end{quote}
We get a similar description of the phenomenology of focal attention, also set within an extramissionist account, but where the dioptric character of vision is made explicit and exploited in an explanation for how we can see many things all at once. For consider a ray sent from a single eye. It would be natural to expect that it sees only that with which it is in contact. But what we see is not restricted in this way. Reflection on binocular vision provides an explanation. The rays from both eyes form a cone. Where the rays intersect is the point of focal attention where things appear exactly in a way that is meant to be consistent with many other things appearing as well if not exactly. Query: Do the many that appear if not exactly confined to the region within the cone? If so, this model does not fit the Augustinian passage where the object seen, if obscurely, is outside of the cone. If the many that appear if not exactly are not confined to the cone, then we so far lack an explanation of how we see the many in terms of the extramissionist dioptrics.

% subsection de_genesi_ad_litteram_libri_duodecim (end)

\subsection{\emph{Sermon 277}} % (fold)
\label{sub:subsection_name}

In 411 AD, on the Birthday of the Martyr Vincent, Augustine delivered the following as part of a sermon:
\begin{quote}
	In this very body, which we carry around with us, I can find something whose inexpressible swiftness astonishes me; the ray from our eye, with which we touch whatever we behold. What you see, after all, is what you touch with the ray from your eye. (\emph{Sermon} 277 10; Edmund Hill, O.P., *The Works of Saint Augustine, A Translation for the 21st Century, Part III, Sermons, Volume 8, 273-305A, On the Saints*, Hyde Park, NY: New City Press, 1994, 38)
\end{quote}
Augustine is using the familiar tactile metaphor associated with extramission theories. But what is it a metaphor for? Arguably, touch is a metaphor for the presentation in sight of the object of vision. That we touch whatever we behold may be too weak, by itself, to establish that, but it is combined with the claim that what you see is what you touch with the visual ray. On this reading, visual presentation is either reduced to or is at the very least modelled on tactile presentation. If that is right, then Hill is wrong to speculate that ``presumably on meeting a visible object <the rays> send back the message to the subject...or perhaps they bounce straight back to the eye like radar'' 46 n.17. The rays, as Hill conceives of them, are merely part of the causal medium through which information about the perceived object is conveyed. But if the visual rays touch the objects of perception, then they are perceived where they are, and thus there is no need for a signal to return to the subject.

Augustine explains occlusion as the obstruction of visual rays. This could not be an argument for the extramission theory, as occlusion is equally well explained on the intromissionist hypothesis. Rather, the example of a man obscuring a distant column, is setting up the real topic of \emph{Sermon} 277 10, the ``inexpressible swiftness'' of the visual rays, which will lead Augustine to an interpretation of Paul's phrase ``in the twinkling of an eye'' (1 Corinthians 15:52) in \emph{Sermon} 277 11, understood as the speed at which the body will be resurrected.

Augustine speaks of the ``inexpressible swiftness'' of the visual rays, but, strictly speaking, their action is instantaneous. If two objects, one near---a man---and one far---a column---are visible to the perceiver, in the circumstances of perception (and hence the man no longer occludes the column), then it is not the case that the visual rays reach the near object sooner than the far: 
\begin{quotation}
	You don't get to him sooner and to it later; and here he is, nearby, and it's a long way off. If you wanted to walk, you would get to the man sooner than to the column; because you wanted to see, you have got to the column as soon as the man. 
	
	And yet, as soon as you open your eyes, lo and behold, you yourself are here, your ray is there. As soo as you wanted to see it, you reached it by seeing it. ... Just opening your eyes constitutes reaching it. (\emph{Sermon} 277 10; Edmund Hill, O.P., *The Works of Saint Augustine, A Translation for the 21st Century, Part III, Sermons, Volume 8, 273-305A, On the Saints*, Hyde Park, NY: New City Press, 1994, 39)
\end{quotation}

Notice that the ``inexpressible swiftness'' of the visual rays is contrasted with the speed of corporeal processes such as walking. Perhaps, like Philoponus, Augustine maintains that instantaneous action at a distance is only possible for incorporeal activity (\emph{In de anima} 325 1-341 9). After all, instantaneous action at a distance would require a body to travel at infinite speed, but bodies, no matter how swift, only travel at finite speeds. 

The contrast that Augustine draws between the speed of the eyelid in opening one's eyes and the speed of the visual rays thereby unleashed is also relevant:
\begin{quote}
	The twinling of an eye does not consist in closing and opening the eyelids, because this is done more slowly than seeing. You bat an eyelid more slowly than you direct a ray. Your ray gets to the sky more quickly than the batted eyelids reach the eyebrow. (\emph{Sermon} 277 11; Edmund Hill, O.P., *The Works of Saint Augustine, A Translation for the 21st Century, Part III, Sermons, Volume 8, 273-305A, On the Saints*, Hyde Park, NY: New City Press, 1994, 39)
\end{quote}
Notice that Augustine identifies the speed of the visual rays with the speed of seeing, providing further evidence for the hypothesis that visual presentation is being understood as a kind of touch by visual rays.


% subsection subsection_name (end)

\subsection{\emph{De Trinitate}} % (fold)
\label{sub:_emph_de_trinitate}

\begin{quote}
	For the mind does not know other minds and not know itself, as the eye of the body sees other eyes and does not see itself; for we see bodies through the eyes of the body, because, unless we are looking into a mirror, we cannot refract and reflect the rays themselves which shine for through the eyes, and touch whatever we discern---a subject, indeed, which is treated of most subtly and obscurely, until it be clearly demonstrated whether the fact be so, or whether it be not. But whatever is the nature of the power by which we discern through the eyes, certainly, whether it be rays or anything else, we cannot discern with the eyes that power itself; but we inquire into it with the mind, and if possible, understand this with the mind. (\emph{De Trinitate} 9 3; Rev Arthur West Haddan, B.D., *On the Trinity, Augustine of Hippo*, Fig, 2012, 238)
\end{quote}
Again, the present extramissionist account is derived neither from scripture nor reason, but is accepted as received authoritative opinion. It is potentially revisable in the way that the deliveries of scripture and reason are not. Notice that Augustine, after having introduced the extramissionist imagery of rays, immediately brackets that commitment, claiming that it is treated subtly and obscurely and claims that the explanation of perceptual discernment by rays has not yet been clearly demonstrated.

% subsection _emph_de_trinitate (end)

% section extramission_in_the_augustinian_emph_corpus (end)








% no cite
\nocite{Tourscher:1933rw}

%Bibliography
\bibliographystyle{plainnat}
\bibliography{Philosophy}

\end{document}